\section{Einleitung}

% \begin{itemize}
%     \item Informationen über die Firma
%     \begin{itemize}
%         \item Geschichte
%         \item Unternehmensstruktur
%         \item Quantitative Daten
%     \end{itemize}
%     \item Arbeitsumfang
%     \item Erwartung für Praktikant
%     \item Aufgabestellung
% \end{itemize}

Mein Praxissemester findet im Rahmen der \glsfirst{FPO} 2008 für den Studiengang Angewandte Informatik B.Sc. und dessen Modulhandbuch \citep{Hochschule_Worms_FPO} statt. Die Stellensuche orientierte sich an dem Schwerpunkt Networks \& Security und spezifischer auf Cybersicherheit und Penetration Testing. 

Da es sich um einen spezifischen Bereich handelt, war meine Suche nach einer geeigneten Firma, an die ich meine Bewerbungsunterlagen schickten, eingeschränkt. Die Firma Wallsec GmbH in Wiesloch-Walldorf hatte eine offene Studentstelle. Da diese Stelle genau meine Ziele entsprach, bewarb ich mich für die Position. Die Aufgabe in der Stellenbeschreibung beziehen sich hauptsächlich um Durchführung von Penetrationstests, um die Analyse von Source-Code und um Evaluation und Entwicklung von Sicherheitsprozessen. Als Voraussetzung verlangte Wallsec  wenig Fachkenntnisse im Bereich Sicherheit und Penetration Testing, aber wollte große Interesse von den Kandidaten Lernbereitschaft \citep{Wallsec}.

Das gesamte Prozess von Bewerbung bist zum Einstig dauerte ungefähr einen Monat. Am 15. Juli 2022 fing ich an, bei Wallsec als Praktikant im Vollzeit zu arbeiten. In diesem Bericht werden wir folgende Themen bearbeiten:

\begin{itemize}
   \item Informationen über die Firma Wallsec
   \item Konzepte von Penetration Testing
   \item Aufgabebereich der Tätigkeit
   \item Ergebnis des Praxissemesters
\end{itemize}

\subsection{Wallsec}
Die Firma wurde im Jahr 2020 von Peter Todorov in Walldorf-Wiesloch, Baden Württemberg, gegründet. Laut der Beschreibung der Webseite fokussierte sie sich auf die Planung, Bereitstellung und Risikoanalysen von Sicherheits-Infrastrukturen von Firmen verschiedenen Größen \citep{Wallsec}. Wallsec bietet folgenden Leistungen an:

\begin{itemize}
   \item Penetrationstests
   \item Schwachstellenmanagement
   \item Richtlinien
   \item Automatisierung
   \item DevOps und CI/CD Pipeline Sicherheit
   \item Beratung im Bereich Cyberabwehr
\end{itemize}








