\section{Einleitung}

% \begin{itemize}
%     \item Informationen über die Firma
%     \begin{itemize}
%         \item Geschichte
%         \item Unternehmensstruktur
%         \item Quantitative Daten
%     \end{itemize}
%     \item Arbeitsumfang
%     \item Erwartung für Praktikant
%     \item Aufgabestellung
% \end{itemize}

Mein Praxissemester findet im Rahmen der \glsfirst{FPO} 2008 für den Studiengang Angewandte Informatik B.Sc. und dessen Modulhandbuch \cite{Hochschule_Worms_FPO} stat. Die Stellensuche orientierte sich auf dem Schwerpunkt Networks \& Security und spezifischer auf Cybersicherheit und Penetration Testing. 

Da es sich um einen spezifischen Bereich geht, war die Suche auf wenigen Firmen eingeschränkt, wo ich meine Bewerbungsunterlagen schickte. Die Firma Wallsec GmbH in Wiesloch-Walldorf hatte eine offene Stelle für Student. Da diese Stelle genau meinen Ziele entsprachen, bewarb ich mich für die Position. Die Aufgabe in der Stellenbeschreibung ging hauptsächlich um Durchführung von Penetrationstests, um Source-Code-Analyse, um Dokumentation Analyse und Erstellen, um Evaluation und Entwicklung von Sicherheitsprozessen. Als Voraussetzung verlangte Wallsec zwar wenig Fachkenntnis im Bereich Sicherheit und Penetration Testing, aber wollte große Interesse von den Kandidaten für das Lernen \cite{Wallsec}.

Das Bewerbungs- bis zum Einstiegsverfahren dauert ungefähr einen Monat. Am 15ten Juli 2022 fing ich an bei Wallsec als Praktikant im Vollzeit zu arbeiten. In diesem Bericht werden wir folgenden Themen bearbeiten:

\begin{itemize}
   \item Informationen über die Wallsec
   \item Konzepte von Penetration Testing
   \item Aufgabebereich der Tätigkeit
   \item Ergebnis des Praxissemesters
\end{itemize}

\subsection{Wallsec}
Die Firma wurde im Jahr 2020 von Peter Todorov in Walldorf-Wiesloch, Baden Württemberg, gegründet. Laut der Beschreibung der Webseite fokussierte sie auf die Planung, Bereitstellung und Risikoanalysen von Sicherheits-Infrastrukturen von Firmen verschiedenen Größen \cite{Wallsec}. Wallsec bietet folgenden Leistungen an:

\begin{itemize}
   \item Penetrationstests
   \item Schwachstellenmanagement
   \item Richtlinien
   \item Automatisierung
   \item DevOps und CI/CD Pipeline Sicherheit
   \item Beratung im Bereich Cyberabwehr
\end{itemize}








