\section{Durchführung der Aufgabe}

In diesem Kapitel beschreiben wir konkret, wie die Arbeit während meines Praxissemesters sich entwickelte. Die ersten zwei Wochen dienten als Einarbeitung und Einstieg. Nach dieser Phasen bekam ich langsam und unter Betreuung mehr Verantwortlichkeit und mehr Freiheit, um die Arbeit durchzuführen. In der folgenden Tabellen wird der Ablauf systematisch und ohne Einzelheit beschrieben. In dem zweiten Teil dieses Kapitels geben wir eine ausführliche Beschreibung eines Projekts.

\begin{table}[H]
    \setstretch{1.0}
    \begin{tabularx}{\textwidth}{|c|X|}
    \toprule
    \multicolumn{2}{c}{\textbf{Auflistung der Aufgabe}} \\
    \midrule
    \multicolumn{1}{c}{\textbf{Woche}} & \multicolumn{1}{c}{\textbf{Aubgabeschreibung}} \\
    \hline
    1 - 2       & \begin{itemize}
                    \item Installation von einer virtuellen Maschine für die Testumgebungen
                    \item Einführung in der Arbeitsablauf der Firma
                    \item Einführung, Installation und Einstellungen von \gls{burp}
                    \item Einführung in einem laufenden Projekt, um über das Ablauf- und Dokumentationsverfahren zu lernen
                    \item Durchführung und Wiederholungen von einigen Tests, um mich an den gegebenen Tools zu gewöhnen
                    \item Teilnahmen an einer Abschlussmeeting des laufenden Projekts, um das Verfahren und den Ablauf des Kundenkontakt zu erkennen und später zu wiederholen
                \end{itemize} \\
        \hline
    3 - 5       & \begin{itemize}
                    \item Start an ein neues Projekt mit einem Meeting mit dem Kunde, wo die zu getestete Anwendung präsentiert wurde
                    \item Definition der Umfang des Tests, wie Anmeldedaten, Rolle der zu getestete Nutzer und Einschränkungen
                    \item Einführung, Installation und Einstellungen von \gls{burp}
                    \item Durchführung von Tests nach einem vorgegeben Checklist
                    \item Dokumentation der durchgeführten Testen und dessen gefundene \glsplural{Schwachstelle}
                    \item Abschlussmeeting mit dem Kunden, um die \glsplural{Schwachstelle} und deren Ausnutzung zu präsentieren und zu demonstrieren
                \end{itemize} \\ 
        \hline
    3 - 5       & \begin{itemize}
                    \item xxxxx 
                \end{itemize} \\ 


       \bottomrule
    \end{tabularx}
\end{table}

\section{Ausführliche Beschreibung eines Penetration Testing innerhalb des Praxissemesters}


