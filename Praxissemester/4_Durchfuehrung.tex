\section{Durchführung der Aufgabe}

In diesem Kapitel beschreiben wir konkret, wie die Arbeit während meines Praxissemesters sich entwickelte. Die ersten zwei Wochen dienten als Einarbeitung und Einstieg. Nach dieser Phasen bekam ich langsam und unter Betreuung mehr Verantwortlichkeit und mehr Freiheit, um die Arbeit durchzuführen. In der folgenden Tabellen wird der Ablauf systematisch und ohne Einzelheit beschrieben. In dem zweiten Teil dieses Kapitels geben wir eine ausführliche Beschreibung eines Projekts.

Jedes Projekt besitzt innerhalb von Wallsec einen festgelegten Aufbau. Dieser kann in den folgenden Punkten zusammengefasst werden:

\begin{enumerate} \label{Projektablauf}
    \item \textit{Kick-off Meeting} mit den Kunden, um grundsätzliche Information über die Anwendung zu bekommen
    \item Definition der Umfang des Tests, wie Anmeldedaten, Rolle der zu getestete Nutzer, \glsplural{Tennant} und Einschränkungen
    \item Durchführung von Tests nach einem vorgegebenen Checklist
    \item Dokumentation der durchgeführten Testen, dessen gefundene \glsplural{Schwachstelle} und Vorschläge zur Härtung der Anwendung
    \item Abschlussmeeting mit dem Kunden, um die \glsplural{Schwachstelle} und deren Ausnutzung zu präsentieren und zu demonstrieren
\end{enumerate} 

\section{Wöchentliche Zusammenfassung meines Praxissemesters}

\begin{table}[H]
    \setstretch{1.0}
    \begin{tabularx}{\textwidth}{|c|X|}
    \toprule
    \multicolumn{2}{c}{\textbf{Auflistung der Aufgabe}} \\
    \midrule
    \multicolumn{1}{c}{\textbf{Woche}} & \multicolumn{1}{c}{\textbf{Aufgabeschreibung}} \\
    \hline
    1 - 2    & Einarbeitung:
                \begin{itemize}
                    \item Installation von einer virtuellen Maschine für die Testumgebungen
                    \item Einführung in der Arbeitsablauf der Firma
                    \item Einführung, Installation und Einstellungen von \gls{burp}
                    \item Einführung in einem laufenden Projekt, um über das Ablauf- und Dokumentationsverfahren zu lernen
                    \item Durchführung und Wiederholungen von einigen Tests, um mich an den gegebenen Tools zu gewöhnen
                    \item Teilnahmen an einer Abschlussmeeting des laufenden Projekts, um das Verfahren und den Ablauf des Kundenkontakt zu erkennen und später zu wiederholen
                \end{itemize} \\
        \hline

    3 - 4       &  Start, Durchführung und Abschluss eines neuen Pentesting-Projekts an einem Versicherungsanwendung mit dem obigen beschriebenen Schritte (\ref{Projektablauf})  \\ 
    
    \hline

    5 - 6       & Weiterarbeitung an der Installation, an der Einstellungen und an der Nutzung der Tools \gls{TheHive} und \gls{Cortex}. Bereitstellung von Skripts zum Herunterladen von statistische Daten der Anwendungen und zur Automatisierung deren Nutzung.  \\ 

    \hline

    7 - 8      &  Start, Durchführung und Abschluss eines neuen Pentesting-Projekts an einem Marketing-Webanwendung mit dem obigen beschriebenen Schritte (\ref{Projektablauf}) \\

    \hline

    9 - 10      &  Start, Durchführung und Abschluss eines neuen Pentesting-Projekts an Netzwerk-Umgebungen mit dem obigen beschriebenen Schritte (\ref{Projektablauf}). Die durchgeführten Tests konzentrieren sich auf die Sicherheit einer Netzwerk. Für dieses Projekt spielt das Tool gls{nmap} eine wichtige Rolle, da das Ziel war, Hosts, Dienst und dessen \gls{Schwachstelle} zu erkennen \\


       \bottomrule
    \end{tabularx}
\end{table}

\section{Ausführliche Beschreibung eines Penetration Testing innerhalb des Praxissemesters}


