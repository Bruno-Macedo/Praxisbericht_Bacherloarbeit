% here goes to the description
\newglossaryentry{CIA2}{
    name=\glslink{CIA}{Confidentiality, Integrity and Availability (\gls{CIA})},
    description={Beschreibt die drei wichtigsten Schutzziele der IT-Sicherheit: Vertraulichkeit, Integrität und Verfügbarkeit \citep{Wendzel_It-Sicherheit} }
}
% here goes to the acronym
\newglossaryentry{CIA}{
    type=\acronymtype,
    name=CIA,
    first={Confidentiality, Integrity and Availability (CIA)\glsadd{CIA2}},
    see=[Glossary:]{\gls{CIA2}}, 
    description=\glslink{CIA2}{Confidentiality, Integrity and Availability}
}
% =====

%=====
\newglossaryentry{NIST2}{
    name=\glslink{NIST}{National Institute of Standards and Technology (\gls{NIST})},
    description={Eine US-Behörden, die für die Reglungen, Vereinheitlichung und Weiterentwicklung im Bereich Informationstechnologie zuständig \citep{Hochschule_Worms_FPO} }
}

\newglossaryentry{NIST}{
    type=\acronymtype,
    name=NIST,
    first={National Institute of Standards and Technology (NIST)\glsadd{NIST2}},
    see=[Glossary:]{\gls{NIST2}}, 
    description=\glslink{NIST2}{National Institute of Standards and Technology}
}
% =====

%=====
\newglossaryentry{CVSS2}{
    name=\glslink{CVSS}{Common Vulnerability Scoring System (\gls{CVSS})},
    description={Internationale Standards für die Bewertung von Verwundbarkeiten von IT-Systemen. Es wurde im Jahr 2005 von dem National Infrastructure Advisory Councils eingeführt und wird heute von dem Forum of Incident Response and Security Teams verwaltet \citep{SecInsider_CVSS} }
}

\newglossaryentry{CVSS}{
    type=\acronymtype,
    name=CVSS,
    first={Common Vulnerability Scoring System (CVSS)\glsadd{CVSS2}},
    see=[Glossary:]{\gls{CVSS2}}, 
    description=\glslink{CVSS2}{Common Vulnerability Scoring System}
}
% =====

%=====
\newglossaryentry{OSINT2}{
    name=\glslink{OSINT}{Open-source intelligence (\gls{OSINT})},
    description={Datenerhebung und Sammlung aus offenen Quellen, wie von Online-Repositories, Nachrichten, sozialen Netzwerken, wissenschaftlichen Texten unter anderen öffentlichen Quellen. In diesem Fall gibt es keine direkten Kontakte mit dem Ziel. Es kann auch passive Reconnaissance genannt werden \citep{Abel_OSINT}}
}

\newglossaryentry{OSINT}{
    type=\acronymtype,
    name=OSINT,
    first={Open-source intelligence (OSINT)\glsadd{OSINT2}},
    see=[Glossary:]{\gls{OSINT2}}, 
    description=\glslink{OSINT2}{Open-source intelligence}
}
% =====

%=====
\newglossaryentry{ROE2}{
    name=\glslink{ROE}{Rules of Engagement (\gls{ROE})},
    description={Bezieht sich auf ein Vertrag, der zwischen Kunden und Tester abgeschlossen wird, um den Umfang und die Rahmenbedingungen des Tests festzulegen. Dieses Dokument enthält unter anderem folgenden Informationen: Umgang mit sensitiven Daten, Notfallkontakten, Identifikation der zu testenden Zielsysteme und Einschränkungen des Testsystem \citep{Triaxiom_ROE}}
}

\newglossaryentry{ROE}{
    type=\acronymtype,
    name=ROE,
    first={Rules of Engagement (ROE)\glsadd{ROE2}},
    see=[Glossary:]{\gls{ROE2}}, 
    description=\glslink{ROE2}{Rules of Engagement}
}
%=====

\newglossaryentry{FPO}{
    type=\acronymtype,
    name=FPO,
    first={Fachspezifische Prüfungsordnung (FPO)\glsadd{FPO}},
    see=[Glossary:]{\gls{FPO}}, 
    description=\glslink{FPO}{Fachspezifische Prüfungsordnung}
}

\newglossaryentry{Cyberangriff} {
    name={Cyberangriff},
    plural={Cyberangriffe},
    description={Angriffe, die über den Cyberspace stattfinden. Solche Angriffe zielen auf Unternehmen und deren Infrastrukturen, um sie zu zerstören, lähmen, kontrollieren oder die Integrität deren Daten zu stehlen oder zu dominieren \citep{NIST_Definitionen}}
}

\newglossaryentry{Websocket} {
    name={Websocket},
    plural={Websockets},
    description={Netzwerkprotokoll auf der Transportschicht, das eine bidirektionale Verbindung zwischen \textit{Client} und \textit{Server} anbietet. Während bei \gls{http2} findet die Verbindung immer auf Basis von \textit{Request} und \textit{Response} statt, bei Websockets wird eine Verbindung hergestellt und Daten werden im Laufzeit dieser Verbindung in beiden Richtung geschickt, ohne dass \textit{Client} und/oder \textit{Server} sich erneut identifizieren müssen \citep{Websockets}}
}

\newglossaryentry{Cybersicherheit} {
    name={Cybersicherheit},
    description={Diese Domäne umfasst Kenntnisse und Methoden für den Schutz, Prävention, Wiederherstellung von elektronischen Kommunikationsmittel und deren Inhalt. Es konzentriert sich auf ihrer Verfügbarkeit, Integrität, Authentizität, Vertraulichkeit und Verbindlichkeit \citep{NIST_Definitionen}}
}

\newglossaryentry{Verwundbarkeit} {
    name={Verwundbarkeit},
    plural={Verwundbarkeiten},
    description={Auch \textit{vulnerability} genannt. Es beschreibt eine von Angreifer ausnutzbare Schwachstelle \citep{Wendzel_It-Sicherheit}}
}

\newglossaryentry{Schwachstelle} {
    name={Schwachstelle},
    plural={Schwachstellen},
    description={Schwäche eines Systems \citep{Wendzel_It-Sicherheit}}
}

\newglossaryentry{Webanwendung} {
    name={Webanwendung},
    plural={Webanwendungen},
    description={Internetseiten, die eine Interaktion ermöglichen. Diese Interaktion kann beispielsweise Login, Einkauf, Erstellung und Manipulation von Daten sein. Die meisten Webanwendungen sind mit einer Datenbank verbunden. Webseiten sind anderseits statische Seiten, deren Inhalt nicht dynamisch aktualisiert ist \citep{Essential_Desigs_Seite_x_Anwendung}}
}

\newglossaryentry{port} {
    name={Port},
    plural={Ports},
    description={ist eine Zahl, die ein Dienst oder eine Verbindung identifiziert. Es geht hier um eine logische Adressierung zur Identifizierung eines oder mehrere Prozessen \citep{porttanenbaum}}
}

\newglossaryentry{Pentester} {
    name={Pentester},
    plural={Pentester},
    description={Auch Ethical Hacker genannt ist ein Sicherheitsanalyst, der sich damit beschäftigt, \glsplural{Schwachstelle}  von IT-Systemen zu finden \citep{pentester}}   
}

\newglossaryentry{burp} {
    name={Burp Suite},
    description={auch Burp genannt ist eine von der Firma PortSwigger in Java-Programmiersprache entwickelte Anwendung für die Durchführung von Sicherheitstests in Webanwendungen. Mit verschiedenen Funktionalitäten unterstützt die Anwendung während allen Phasen eines Penetration Testings von Reconnaissance bis zum Angriff \citep{burp}}   
}

\newglossaryentry{Tenant} {
    name={Tenant},
    plural={Tenants},
    description={Einige Webanwendungen werden so konzipiert, dass verschiedene unabhängige Gruppen sie verwenden können. Als Beispiel kann eine Plattform für Online-Shop von verschiedenen Anbietern benutzt werden. Obwohl jeder Anbieter seine eigenen Namen, Marken und Produkte hat, benutzen alle nur eine Plattform. Jeder von diesem Anbieter nennen wir Tenants}   
}

\newglossaryentry{TheHive} {
    name={TheHive Project},
    description={TheHive ist eine Open Source Plattform für die Verwaltung und Weiterbearbeitung von Sicherheitsvorfällen. Es integriert andere Plattformen und Anwendungen, wie Cortex, um Informationen und Handlungen bereitzustellen, damit die Arbeit vom Security Operation Center auf einer Plattform konzentriert bleibt \citep{TheHive}}   
}

\newglossaryentry{Cortex} {
    name={Cortex},
    description={Wie \gls{TheHive} ist, Cortex auch eine Open Source Plattform für die Verwaltung und Weiterbearbeitung von Sicherheitsvorfällen. Es funktioniert wie eine Analysis Engine, die Informationen sammelt und Antworten/Aktionen je nach Fälle durchführt. Es kann eigenständig oder integriert mit TheHive funktionieren \citep{TheHive}}   
}

\newglossaryentry{nmap} {
    name={nmap},
    description={(Network Mapper) ist eine Open-Source Anwendung für die Netzwerkanalyse. Mit diesem Tool ist es möglich, Hosts, Diensten (und deren Versionen) schnell
    zu entdecken \citep{nmap}}    
}

\newglossaryentry{scout} {
    name={Scout Suite},
    description={Audi-Tool für Sicherheitsüberprüfungen von Cloud-Umgebungen. Mit dem Tool werden Einstellungsinformationen gelesen und in einem lesbaren Dateiformat ausgegeben \citep{scoutsuite}}   
}

\newglossaryentry{dirb} {
    name={Dirbuster, Gobuster, usw.},
    description={sind Anwendungen, die mit Brute Force, versuchen, Dateien und Verzeichnisse innerhalb Webanwendungen zu finden \citep{kalitools}}   
}

\newglossaryentry{javascript} {
    name={Javascript},
    description={ist eine Programmiersprache, die in Webanwendungen verwendet wird, um komplexe Strukturen wie Animationen, Bilder, Ton und Interaktionen zu implementieren \citep{javascriptdefinition}}   
}

\newglossaryentry{python} {
    name={Python},
    description={ist eine im Jahr 1991 entwickelte Programmiersprache, die Flexibilität, Lesbarkeit und Wiederverwendung anbieten sollte. Python stellt verschiedene Module zur Verfügung, die sich leicht an anderen Anwendungen anpassen sollen \citep{python}}   
}

\newglossaryentry{Skript} {
    name={Skript},
    plural={Skripte},
    description={in dem Informatikbereich bezieht sich meistens auf eine Textdatei, die Kommandos in einer Programmiersprache, wie Python, Javascript, Bash oder anderen, beinhalt. Sie dienen dazu, Aufgabe zu automatisieren, die sonst manuell durchgeführt werden sollten \citep{skript}}   
}

\newglossaryentry{graphql} {
    name={GraphQL},
    description={eine von Facebook entwickelte Sprache für die Kommunikation zwischen Anwendungen, um die genau angefragte Information zu bekommen \citep{GraphQL}}   
}

%=====
\newglossaryentry{OWASP2}{
    name=\glslink{OWASP}{Open Web Application Security Project\textregistered (\gls{OWASP})},
    description={Eine Non-Profit Organisation, die sich darauf fokussiert, die Sicherheit in dem Umgang mit Webanwendungen zu gewährleisten. Die Organisation verteilt Open-Source Informationen über sichere Entwicklung, Dokumentation, Best-Practices zu dem sicheren Umgang mit dem Internet und Bildung \citep{Triaxiom_ROE}}
}

\newglossaryentry{OWASP}{
    type=\acronymtype,
    name=OWASP\textregistered,
    first={Open Web Application Security Project\textregistered (OWASP\textregistered)\glsadd{OWASP2}},
    see=[Glossary:]{\gls{OWASP2}}, 
    description=\glslink{OWASP2}{Open Web Application Security Project\textregistered}
}
%=====

%=====
\newglossaryentry{dvwa2}{
    name=\glslink{dvwa}{Damn Vulnerable Web Application  (\gls{dvwa})},
    description={ist eine mit Schwachstellen absichtlich entwickelte Webanwendung für Test- und Lernumgebung. Sie wird meist von Entwicklern verwendet, um sich mit Schwachstellen und Best-Practices zu kennen. Dieses Plattform hat als Umfang die meisten bekannten Web-Angriffe  \citep{DVWA} }
}

\newglossaryentry{dvwa}{
    type=\acronymtype,
    name=DVWA,
    first={Damn Vulnerable Web Application (DVWA)\glsadd{dvwa2}},
    see=[Glossary:]{\gls{dvwa2}}, 
    description=\glslink{dvwa2}{Damn Vulnerable Web Application }
}
% =====

%=====
\newglossaryentry{xss2}{
    name=\glslink{xss}{Cross-Site Scripting (\gls{xss})},
    description={ist ein Angriff, wo bösartige Code in eine Webanwendung absichtlich hinzugefügt wird, um an Anmeldedaten oder Sitzungsinformationen zu gelangen. In diesem Fall ist das Ziel des Angriffes, eine legitime Nutzung der Anwendung vorzutäuschen, um Informationen zu stehlen (wie Passwörter, persönliche oder finanzielle Daten) oder die Anwendung zu beschädigen \citep{xssliteratur} }
}

\newglossaryentry{xss}{
    type=\acronymtype,
    name=XSS,
    first={Cross-Site Scripting (XSS)\glsadd{xss2}},
    see=[Glossary:]{\gls{xss2}}, 
    description=\glslink{xss2}{Cross-Site Scripting}
}
% =====


%=====
\newglossaryentry{pof2}{
    name=\glslink{pof}{Proof of Concept (\gls{pof})},
    description={ist eine Demonstration, dass eine Methode funktioniert. In dem Sicherheitsbereich zeigt es, dass eine Schwachstelle ausnutzbar ist. \citep{proofofconcept}}
}

\newglossaryentry{pof}{
    type=\acronymtype,
    name=PoF,
    first={Proof of Concept (PoF)\glsadd{pof2}},
    see=[Glossary:]{\gls{pof2}}, 
    description=\glslink{pof2}{Proof of Concept}
}

%=====
\newglossaryentry{http2}{
    name=\glslink{http}{Hypertext Transfer Protocol (\gls{http})},
    description={ein 1989 entwickelter Model für die Übertragung von Dateien in dem Internet. Die neue Version, HTTP/2, sollte mehr Sicherheit und Leistung mit wenigen Datennutzung anbieten \citep{http_protocol}}
}

\newglossaryentry{http}{
    type=\acronymtype,
    name=HTTP,
    first={Hypertext Transfer Protocol (HTTP)\glsadd{http2}},
    see=[Glossary:]{\gls{http2}}, 
    description=\glslink{http2}{Hypertext Transfer Protocol}
}

% \newglossaryentry{Schwachstelle} {
%     name={Schwachstelle},
%     plural={Schwachstellen}
%     description={Schwäche eines Systems \citep{Wendzel_It-Sicherheit}}
% }
    