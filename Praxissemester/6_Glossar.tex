% here goes to the description
\newglossaryentry{CIA2}{
    name=\glslink{CIA}{Confidentiality, Integrity and Availability (\gls{CIA})},
    description={Beschreibt die drei wichtigsten Schutzziele der IT-Sicherheit, und zwar Vertraulichkeit, Integrität und Verfügbarkeit \cite{Wendzel_It-Sicherheit} }
}
% here goes to the acronym
\newglossaryentry{CIA}{
    type=\acronymtype,
    name=CIA,
    first={Confidentiality, Integrity and Availability (CIA)\glsadd{CIA2}},
    see=[Glossary:]{\gls{CIA2}}, 
    description=\glslink{CIA2}{Confidentiality, Integrity and Availability}
}
% =====

%=====
\newglossaryentry{NIST2}{
    name=\glslink{NIST}{National Institute of Standards and Technology (\gls{NIST})},
    description={USA-Behörden dafür zuständig, Reglungen im Bereich Informationstechnologie zu vereinheitlichen und voranzutreiben \cite{Hochschule_Worms_FPO} }
}

\newglossaryentry{NIST}{
    type=\acronymtype,
    name=NIST,
    first={National Institute of Standards and Technology (NIST)\glsadd{NIST2}},
    see=[Glossary:]{\gls{NIST2}}, 
    description=\glslink{NIST2}{National Institute of Standards and Technology}
}
% =====

%=====
\newglossaryentry{CVSS2}{
    name=\glslink{CVSS}{Common Vulnerability Scoring System (\gls{CVSS})},
    description={Internationale Standards für die Bewertung von Verwundbarkeiten von IT-Systemen. Es wurde im Jahr 2005 von dem National Infrastructure Advisory Councils entstanden und ist heute von dem Forum of Incident Response and Security Teams  verwaltet \cite{SecInsider_CVSS} }
}

\newglossaryentry{CVSS}{
    type=\acronymtype,
    name=CVSS,
    first={Common Vulnerability Scoring System (CVSS)\glsadd{CVSS2}},
    see=[Glossary:]{\gls{CVSS2}}, 
    description=\glslink{CVSS2}{Common Vulnerability Scoring System}
}
% =====

%=====
\newglossaryentry{OSINT2}{
    name=\glslink{OSINT}{Open-source intelligence (\gls{OSINT})},
    description={Datenerhebung und Sammlung aus offenen Quellen, wie von Online-Repositories, Nachrichten, sozialen Netzwerken, wissenschaftliche Texten unter anderen öffentlichen Quellen. In diesem Fall gibt es keine direkte Kontakte mit dem Ziel. Es kann passive Reconnaissance genannt werden \cite{Abel_OSINT}}
}

\newglossaryentry{OSINT}{
    type=\acronymtype,
    name=OSINT,
    first={Open-source intelligence (OSINT)\glsadd{OSINT2}},
    see=[Glossary:]{\gls{OSINT2}}, 
    description=\glslink{OSINT2}{Open-source intelligence}
}
% =====

%=====
\newglossaryentry{ROE2}{
    name=\glslink{ROE}{Rules of Engagement (\gls{ROE})},
    description={Bezieht sich auf ein vertragliches Dokument, der zwischen Kunden und Tester geschlossen wird, um den Umfang und die Rahmenbedingungen des Testes festzulegen. In diesem Dokument steht under anderen folgenden Informationen: Umgang mit sensitiven Daten, Notfallkontakten, Identifikation der zu testenden Objekten und Einschränkungen des Testobjekte \cite{Triaxiom_ROE}}
}

\newglossaryentry{ROE}{
    type=\acronymtype,
    name=ROE,
    first={Rules of Engagemente (ROE)\glsadd{ROE2}},
    see=[Glossary:]{\gls{ROE2}}, 
    description=\glslink{ROE2}{Rules of Engagement}
}
%=====

\newglossaryentry{FPO}{
    type=\acronymtype,
    name=FPO,
    first={Fachspezifische Prüfungsordnung (FPO)\glsadd{FPO}},
    see=[Glossary:]{\gls{FPO}}, 
    description=\glslink{FPO}{Fachspezifische Prüfungsordnung}
}

\newglossaryentry{Cyberangriff} {
    name={Cyberangriff},
    plural={Cyberangriffe},
    description={Angriffe, die über den Cyberspace stattfinden. Solche Angriffe zielen Unternehmen und deren Infrastrukturen, um sie zu zerstören, sie zu lähmen, sie zu kontrollieren oder die Integrität deren Daten zu stehlen oder zu dominieren \cite{NIST_Definitionen}}
}

\newglossaryentry{Cybersicherheit} {
    name={Cybersicherheit},
    description={Dieser Domäne umfasst Kenntnisse und Methode für den Schutz, für die Prävention, für die Wiederherstellung von elektronischen Kommunikationsmittel und dessen Inhalt. Es konzentriert sich in ihrer Verfügbarkeit, Integrität, Authentizität, Vertraulichkeit und Verbindlichkeit \cite{NIST_Definitionen}}
}

\newglossaryentry{Verwundbarkeit} {
    name={Verwundbarkeit},
    plural={Verwundbarkeiten},
    description={Oder als \textit{vulnerability} gekannt. Es beschreibt eine von Angreifer ausnutzbare Schwachstelle \cite{Wendzel_It-Sicherheit}}
}

\newglossaryentry{Schwachstelle} {
    name={Schwachstelle},
    plural={Schwachstellen},
    description={Schwäche eines Systems \cite{Wendzel_It-Sicherheit}}
}

\newglossaryentry{Webanwendung} {
    name={Webanwendung},
    plural={Webanwendungen},
    description={Internetseite, die eine Interaktion ermöglichen. Diese Interaktion kann beispielsweise Login, Einkauf, Erstellung und Manipulation von Daten. Die meisten Webanwendungen sind an einem Datenbank verbunden. Webseite sind seinerseits statische Seite, dessen Inhalt nicht dynamisch aktualisiert sind \cite{Essential_Desigs_Seite_x_Anwendung}}
}

\newglossaryentry{Pentester} {
    name={Pentester},
    plural={Pentester},
    description={Auch Ethical Hacker genannt ist ein Sicherheitsanalyst, der sich damit beschäftigt, \glsplural{Schwachstelle}  von IT-Systemen zu finden\cite{pentester}}   
}

\newglossaryentry{burp} {
    name={Burp Suite},
    description={Oder nur Burp genannt ist eine von der Firma PortSwigger in Java-Programmiersprache entwickelte Anwendungen dafür geeignet, Sicherheitstests durchzuführen. Mit verschiedenen Modulen unterstützt Anwendungen unterstützt während allen Phasen eines Penetration Testings von Reconnaissance bis zum Angriff \cite{burp}}   
}

\newglossaryentry{Tenant} {
    name={Burp Suite},
    plural={Tenants},
    description={Einige Webanwendungen werden so konzipiert, dass verschiedene unabhängige Gruppe verwenden können. Z.B. ein Plattform für Online-Shop kann von verschiedenen Anbieter benutzt werden. Obwohl jeder Anbieter seine eigenen Namen, Marken, Produkte haben, benutzen beide nur einen Plattform. Jeder von diesem Anbieter nennen wir Tenants}   
}

\newglossaryentry{TheHive} {
    name={TheHive Project},
    description={TheHive ist ein Open Source Plattform für die Verwaltung und Weiterarbeitung von Sicherheitsvorfälle. Es integriert andere Plattform und Anwendungen, wie Cortex, um Informationen und Handlungen bereitzustellen, damit die Arbeit von Security Operations Center auf einem Plattform konzentriert bleibt \cite{TheHive}}   
}

\newglossaryentry{Cortex} {
    name={Cortex},
    description={Wie \gls{TheHive}, Cortex ist auch Open Source Plattform für die Verwaltung und Weiterarbeitung von Sicherheitsvorfälle. Es funktioniert wie eine Analysis Engine, die Informationen sammelt und Antworten/Aktionen je nach Fälle durchführt. Es kann allein oder integriert mit TheHive funktionieren \cite{TheHive}}   
}

%=====
\newglossaryentry{OWASP2}{
    name=\glslink{OWASP}{Open Web Application Security Project\textregistered (\gls{OWASP})},
    description={Non-Profit Organisation, die sich darauf fokussiert, die Sicherheit in dem Umgang mit Webanwendungen zu gewährleisten. Die Organisation verteilt Open-Source Informationen über sichere Entwicklung, Dokumentation, Best-Practices zu dem sicheren Umgang in dem Internet und Bildung. \cite{Triaxiom_ROE}}
}

\newglossaryentry{OWASP}{
    type=\acronymtype,
    name=OWASP\textregistered,
    first={Open Web Application Security Project\textregistered (OWASP\textregistered)\glsadd{OWASP2}},
    see=[Glossary:]{\gls{OWASP2}}, 
    description=\glslink{OWASP2}{Open Web Application Security Project\textregistered}
}
%=====

% \newglossaryentry{Schwachstelle} {
%     name={Schwachstelle},
%     plural={Schwachstellen}
%     description={Schwäche eines Systems \cite{Wendzel_It-Sicherheit}}
% }
    