\section{Anwendungsdomäne}

Oft gibt es Nachrichten über Firmen oder Regierungen, deren Geheimnisse im Netz von Angreifern veröffentlicht wurden oder deren Dienste aufgrund eines \gls{Cyberangriff}es unerreichbar sind. Da solche Situationen immer öfter vorkommen, ist das Interesse und die Forschung in dem Bereich \gls{Cybersicherheit} in den letzten Jahren rasant gestiegen \citep{Tanembaum_MBS}. 

Diese Angriffe verletzen die drei wichtigsten Ziele der \gls{Cybersicherheit}: die Vertraulichkeit, die Integrität und die Verfügbarkeit (aus dem Englisch \glsfirst{CIA2}). Diese Zielen werden in den Fachliteraturen wie folgt beschrieben: Schutz gegen unautorisierte Informationsgewinnung; Schutz gegen unautorisierte Datenmanipulation und Zugriffsgewährleistung für authentifizierte und autorisierte Subjekte \citep{Wendzel_It-Sicherheit}. Ein Angriff zielt auf eine \gls{Schwachstelle}n oder auf die \gls{Verwundbarkeit}en eines Systems. Diese könnten dann zu einer Bedrohung werden.

\subsection{Theorie über Penetration Testing}

Schwachstellenanalyse und Penetration Testing sind heutzutage eine oft verwendete Methoden, um \gls{Verwundbarkeit}en zu finden und um zu analysieren. Der \glsfirst{NIST2} beschreibt das Erstere als systematische Analyse eines Systems oder Produktes in Bezug auf ihre Sicherheit, um dessen \gls{Schwachstelle} zu finden; und das Letzere als Methode für die Verifizierung von binärischen Komponenten oder Anwendungen im Ganzen, um zu rauszufinden, ob dessen \glsplural{Verwundbarkeit} in Bezug auf ihre Daten oder Resourcen ausnutzbar sind \citep{NIST_Definitionen}. Die Schwachstellenanalyse umfasst auch eine Datenerhebung, die später im Penetration Testing auf eine autorisierte Weise ausgenutzt wird\citep{Goel_VulAsses_PenTest}. Das Ergebnis dieser zwei Prozesse wird später dem Auftraggeber bekannt gemacht, damit Sicherheitsmaßnahmen umgesetzt werden können.

Der Begriff ist auch als ``Ethical Hacking'', ``Pentest'' oder als ``White Hats (weiße Hüte)'' bekannt. Er umfasst spezifische Analyse von Drohungen und von \glsplural{Verwundbarkeit} eines Produktes und es findet im Rahmen eines Vertrags oder \gls{ROE} zwischen einem Kunden und der Firma oder Person statt, die für die Tests verantwortlich ist \citep{Bishop_PenTest}.

\subsection{Phase und Methodologie eines Penetration Testing}

Ein Penetration Testing findet in einer systematischen Reihenfolge mit drei Hauptphasen statt: Vorbereitung, Implementierung und Analyse \citep{Hessa_study_pentesting}. Während der Vorbereitung werden der Umfang, die Ziele und die Dauer definiert. Bei der Implementierung wird das System oder das Produkt nach Schwachstellen untersucht und diese werden ausgenutzt (\textit{exploited}). In der letzten Phase werden die gefundenen \glsplural{Verwundbarkeit} zusammen mit sämtlichen Vorschlägen zu mitigierenden Maßnahmen dem BAuftraggeber mitgeteilt. In manchen Fällen kann das getestete Zielsystem bezüglich seiner Sicherheit mithilfe der \glsfirst{CVSS} bewertet werden. Dieser Punktmechanismus stellt eine international anerkannte Evaluierung Zielsystems dar. 

Es gibt drei bekannte Methodologien, wie ein Penetration Testing stattfindet: \textit{white box}, \textit{black box} oder \textit{zero-knowledge} und \textit{grey box}. In der ersten Methodologie bekommen die Tester ausführliche Informationen über das getestete Zielsystem, wie Quellcode, interne Logik und Struktur. In der zweiten bekommen die Tester nur \glsfirst{OSINT} Informationen. Die dritte Variante ist eine Mischung aus den ersten zwei, in diesem Fall bekommen die Tester beschränkte Informationen über das zu testende Zielsystem \citep{Ehmer_methoden_testen}.

% \subsection{Tools}

% Es gibt verschiedene Tools, die dazu beitragen ein Objekt in seinem \gls{Verwundbarkeit} zu erkennen (Reconnaissance) und auszunutzen. Hier beschreiben wir die meisten bekannten und verwendet. Zuerst sprechen wir die Reconnaissance-Tools anschließen erklären wir über \textit{exploit-tools}.

%jetzt hier über burp, dirbuster, hydra, nmap, metasploit schreiben.

\subsection{Penetration Testing in \glsplural{Webanwendung}}

\glsplural{Webanwendung} bieten ihren Nutzern eine dynamische und interaktive Umgebung, ohne dass man Anwendungen auf dem eigenen Rechner installieren muss. Im Gegensatz zu Desktop-Anwendungen erlauben \glsplural{Webanwendung} den gleichzeitigen Zugriff und Verwendung mehrerer Nutzer. Der Zugriff findet über verschiedene Plattformen statt, wie Handys, Desktop, Tablet und Laptops. \glsplural{Webanwendung} bieten auch günstigere und schnellere Wartungsmaßnahmen an, da die Hardwarekonfiguration nicht ständig aktualisiert werden muss\citep{webapp}. 

Da die Häufigkeit der Transaktionen mit \glsplural{Webanwendung} ständig steigt, müssen Anbieter die Sicherheit dieser Anwendungen gewährleisten. Die \glsfirst{OWASP} beschäftigt sich damit, die Sicherheit in \glsplural{Webanwendung} zu erforschen. Die Publikationen von der Organisation werden weltweit von Sicherheitsfirmen, Entwicklern und \glsplural{Pentester}n verwendet, um die Tests durchzuführen. 

Jährlich veröffentlicht \gls{OWASP} eine Liste mit den zehn häufigsten Angriffen in \glsplural{Webanwendung} und sicheren Maßnahmen, sie zu vermeiden. Die Organisation bietet auch eine eigene \textit{Security Testing Guide} an, die die Arbeit von Penetration Testern unterstützt, um die \glsplural{Schwachstelle} von Anwendungen zu finden, zu überprüfen und zu härten.

Innerhalb meines Praxissemesters spielten die Publikationen von der \gls{OWASP} eine wichtige Rolle, um spezifische Kenntnisse zu erwerben und um meine Arbeitsweise an die heutigen Anforderungen anzupassen. 