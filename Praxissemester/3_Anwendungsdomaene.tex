\section{Anwendungsdomäne}

Oft bekommen wir Nachrichten über Firmen oder Regierungen, deren Geheimnis im Netz von Angreifer veröffentlicht wurden oder deren Diensten wegen \gls{Cyberangriff} unerreichbar sind. Da solche Situationen öfter als je vorgekommen sind, ist das Interesse und die Forschung in dem Bereich \gls{Cybersicherheit} in den letzten zehn Jahr rasant gestiegen \cite{opvenvpn}. 

Solche Situationen verletzen die drei wichtigsten Ziele der \gls{Cybersicherheit}, und zwar die Vertraulichkeit, die Integrität und die Verfügbarkeit (aus dem Englisch \glsfirst{CIA2}) . Diese Zielen bekommen in den Fachliteraturen folgenden Beschreibungen: Schutz gegen unautorisierte Informationsgewinnung; Schutz gegen unautorisierte Datenmanipulation und Zugriffsgewährleistung für authentifizierte und autorisierte Subjekten \cite{Wendzel_It-Sicherheit}. Ein Angriff zielt \gls{Schwachstelle} oder \gls{Verwundbarkeit} eines Systems. Diese wird dann zu einer Bedrohung, wenn es möglich ist, dieses System auszunutzen.

\subsection{Theorie über Penetration Testing}

Eine heutzutage sehr verwendete Methode, um \gls{Verwundbarkeit} zu finden und zu analysieren ist durch Schwachstellenanalyse und Penetrations Testing. Der \glsfirst{NIST2} beschreibt das erste als systematische Analyse eines Systems oder Produktes in Bezug auf ihrer Sicherheit, um dessen \gls{Schwachstelle} zu finden; und das zweite als Methode für die Verifizierung von binärischen Komponenten oder Anwendungen im Ganzen, um zu finden, ob dessen \glsplural{Verwundbarkeit} in Bezug auf ihre Daten oder Resources ausnutzbare sind \cite{NIST_Definitionen}. Man kann auch sagen, dass es bei der Schwachstellenanalyse eine Datenerhebung stattfindet, die später in der Penetration Testing in eine autorisierte Weise ausgenutzt wird\cite{Goel_VulAsses_PenTest}. Das Ergebnis dieser zwei Prozessen werden später dem Beauftragter bekanntgemacht, damit Sicherheitsmaßnahmen genommen werden können.

Der Begriff ist auch als ``Red Teaming'', als ``Ethikal Hacking'', als ``Pentest'' oder als ``white hats (weiße Huts)'' bekannt. Es umfasst spezifische Analyse von Drohungen und von \glsplural{Verwundbarkeit} eines Produktes und es findet im Rahmen einen Vertrag zwischen einen Kunde und die Firma oder Person statt, die für die Tests verantwortlich sind \cite{Bishop_PenTest}.

\subsection{Phase und Methodologie eines Penetration Testing}

Ein Penetration Testing findet in einer systematischen Reihenfolge mit drei Hauptphasen: Vorbereitung, Implementation und Analysis \cite{Hessa_study_pentesting}. Während der Vorbereitung werden der Umfang, die Ziele und due Dauer definiert. Bei der Implementation wird das System oder das Produkt in ihren Aufbau erkannt, analysiert und ausgenutzt (\textit{exploited}). In der letzten Phase werden die gefundenen \glsplural{Verwundbarkeit} sämtliche Lösungsvorschlägen dem Beauftragter mitgeteilt. In manche Fälle kann das getestete Objekt bezüglich seiner Sicherheit mithilfe der \glsfirst{CVSS} bewertet werden. Diese Punktmechanismus stellt eine internationale anerkannte Evaluation eines Objekts dar. 

Es gibt drei bekannte Methodologie, wo ein Penetration Testing stattfindet, und zwar \textit{white box}, \textit{black box} oder \textit{zero-knowledge} und \textit{grey box}. In der ersten Methodologie bekommen die Tester ausführliche Informationen über das getestet Objekt, wie Quellcode, interne Logik und Struktur. In der zweiten haben die Tester nur \glsfirst{OSINT} Informationen. Die dritte Variante ist eine Mischung aus den ersten zwei, in diesem Fall bekommen die Tester beschränkte Informationen über das zu testende Objekt \cite{Ehmer_methoden_testen}.

\subsection{Tools}

Es gibt verschiedene Tools, die dazu beitragen ein Objekt in seinem \gls{Verwundbarkeit} zu erkennen (Reconnaissance) und auszunutzen. Hier beschreiben wir die meisten bekannten und verwendet. Zuerst sprechen wir die Reconnaissance-Tools anschließen erklären wir über \textit{exploit-tools}.



jetzt hier über burp, dirbuster, hydra, nmap, metasploit schreiben.