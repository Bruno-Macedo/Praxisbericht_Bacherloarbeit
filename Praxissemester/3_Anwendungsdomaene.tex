\section{Anwendungsdomäne}

Oft gibt es Nachrichten über Firmen oder Regierungen, deren Geheimnisse im Netz von Angreifer veröffentlicht wurden oder deren Dienste wegen \gls{Cyberangriff} unerreichbar sind. Da solche Situationen öfter als je vorkommen, ist das Interesse und die Forschung in dem Bereich \gls{Cybersicherheit} in den letzten Jahren rasant gestiegen \citep{Tanembaum_MBS}. 

Diese Angriffe verletzen die drei wichtigsten Ziele der \gls{Cybersicherheit}: die Vertraulichkeit, die Integrität und die Verfügbarkeit (aus dem Englisch \glsfirst{CIA2}). Diese Zielen werden in den Fachliteraturen folgend beschreiben: Schutz gegen unautorisierte Informationsgewinnung; Schutz gegen unautorisierte Datenmanipulation und Zugriffsgewährleistung für authentifizierte und autorisierte Subjekte \citep{Wendzel_It-Sicherheit}. Ein Angriff zielt auf \gls{Schwachstelle} oder auf \gls{Verwundbarkeit} eines Systems. Diese wird dann zu einer Bedrohung, wenn es möglich ist, dieses System auszunutzen.

\subsection{Theorie über Penetration Testing}

Schwachstellenanalyse und Penetration Testing sind heutzutage oft verwendete Methode, um \gls{Verwundbarkeit} zu finden und zu analysieren. Der \glsfirst{NIST2} beschreibt das Erste als systematische Analyse eines Systems oder Produktes in Bezug auf ihrer Sicherheit, um dessen \gls{Schwachstelle} zu finden; und das Zweite als Methode für die Verifizierung von binärischen Komponenten oder Anwendungen im Ganzen, um zu finden, ob dessen \glsplural{Verwundbarkeit} in Bezug auf ihre Daten oder Resources ausnutzbar sind \citep{NIST_Definitionen}. Die Schwachstellenanalyse umfasst auch eine Datenerhebung, die später in der Penetration Testing auf eine autorisierte Weise ausgenutzt wird\citep{Goel_VulAsses_PenTest}. Das Ergebnis dieser zwei Prozessen werden später dem Beauftragten bekanntgemacht, damit Sicherheitsmaßnahmen umgesetzt werden können.

Der Begriff ist auch als ``Ethikal Hacking'', als ``Pentest'' oder als ``white hats (weiße Huts)'' bekannt. Es umfasst spezifische Analyse von Drohungen und von \glsplural{Verwundbarkeit} eines Produktes und es findet im Rahmen eines Vertrags oder \gls{ROE} zwischen einem Kunde und der Firma oder Person statt, die für die Tests verantwortlich sind \citep{Bishop_PenTest}.

\subsection{Phase und Methodologie eines Penetration Testing}

Ein Penetration Testing findet in einer systematischen Reihenfolge mit drei Hauptphasen statt: Vorbereitung, Implementation und Analysis \citep{Hessa_study_pentesting}. Während der Vorbereitung werden der Umfang, die Ziele und die Dauer definiert. Bei der Implementation wird das System oder das Produkt in ihrem Aufbau erkannt, analysiert und ausgenutzt (\textit{exploited}). In der letzten Phase werden die gefundenen \glsplural{Verwundbarkeit} sämtliche Lösungsvorschläge dem Beauftragten mitgeteilt. In manchen Fällen kann das getestete Objekt bezüglich seiner Sicherheit mithilfe der \glsfirst{CVSS} bewertet werden. Dieses Punktmechanismus stellt eine international anerkannte Evaluation eines Objekts dar. 

Es gibt drei bekannte Methodologie, wo ein Penetration Testing stattfindet: \textit{white box}, \textit{black box} oder \textit{zero-knowledge} und \textit{grey box}. In der ersten Methodologie bekommen die Tester ausführliche Informationen über das getestete Objekt, wie Quellcode, interne Logik und Struktur. In der zweiten bekommen die Tester nur \glsfirst{OSINT} Informationen. Die dritte Variante ist eine Mischung aus den ersten zwei, in diesem Fall bekommen die Tester beschränkte Informationen über das zu testende Objekt \citep{Ehmer_methoden_testen}.

% \subsection{Tools}

% Es gibt verschiedene Tools, die dazu beitragen ein Objekt in seinem \gls{Verwundbarkeit} zu erkennen (Reconnaissance) und auszunutzen. Hier beschreiben wir die meisten bekannten und verwendet. Zuerst sprechen wir die Reconnaissance-Tools anschließen erklären wir über \textit{exploit-tools}.

%jetzt hier über burp, dirbuster, hydra, nmap, metasploit schreiben.

\subsection{Penetration Testing in \glsplural{Webanwendung}}

\glsplural{Webanwendung} bieten ihren Nutzer eine dynamische und interaktive Umgebung, ohne dass man Anwendungen in dem eigenen Rechner installieren muss. Im Vergleich zum Desktop-Anwendungen erlauben \glsplural{Webanwendung}, dass mehrere Nutzer die Anwendungen gleichzeitig benutzen können. Der  Zugriff findet über verschiedene Plattformen statt, wie Handys, Desktop, Tablet und Laptops. Sie bietet auch günstigere und schnellere Wartungsmaßnahmen an, da die Hardwarekonfiguration nicht ständig aktualisiert werden muss\citep{webapp}. 

Da die Häufigkeit der Transaktionen mit \glsplural{Webanwendung} ständig steigt, müssen Anbieter die Sicherheit dieser Anwendungen gewährleisten. Die \glsfirst{OWASP} kümmert sich darum, um die Sicherheit in \glsplural{Webanwendung} zu recherchieren. Die Publikationen von der Organisation werden weltweit von Sicherheitsfirmen, Entwickler und \glsplural{Pentester} verwendet, um die Tests durchzuführen. 

Jährlich veröffentlicht \gls{OWASP} eine Liste mit den zehn häufigsten Angriffen in \glsplural{Webanwendung} und sichere Maßnahmen, um sie zu vermeiden. Die Organisation bietet auch eine eigene \textit{Security Testing Guide} an, die die Arbeit von Penetration Testing unterstützt, um die \glsplural{Schwachstelle} von Anwendungen zu finden, zu überprüfen und zu härten.

Innerhalb meines Praxissemesters spielten die Publikationen von der Organisation eine wichtige Rolle, um spezifische Kenntnisse zu erwerben und um meine Arbeitsweise an dem heutigen Anforderungen anzupassen. 