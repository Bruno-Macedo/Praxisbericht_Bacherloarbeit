\documentclass[pdftex, a4paper]{scrartcl}
\usepackage{ngerman}
\usepackage[utf8]{inputenc}
\usepackage[T1]{fontenc}
%\usepackage{array}
%\usepackage{subfiles}
\usepackage{url}
\usepackage[hidelinks]{hyperref}
\usepackage{parskip}
\setlength{\parskip}{1em}
\usepackage{xurl}
\usepackage[nottoc,notlot,notlof]{tocbibind}
\usepackage{longtable}
\usepackage{setspace}
\linespread{1.5}
% \usepackage[
% top    = 2.75cm,
% bottom = 2.50cm,
% left   = 3.00cm,
% right  = 2.50cm]{geometry}
\usepackage{etoolbox}
\AtBeginEnvironment{thebibliography}{\linespread{1}\selectfont}
\usepackage{endnotes}
\interfootnotelinepenalty=10000
\usepackage[acronym,toc,sort=def]{glossaries} 
\usepackage[titletoc]{appendix}
\usepackage{booktabs}
\usepackage{tabularx}
\usepackage{graphicx}
\usepackage{textcomp}
\usepackage{lscape}
\usepackage{fancyhdr} 
\usepackage{multirow}
\usepackage{caption}
\usepackage{natbib}
\usepackage[table,xcdraw]{xcolor}
\usepackage{float}
%\usepackage{lineno}
\usepackage{longtable}
\usepackage{tabularx}
\renewcommand\tabularxcolumn[1]{m{#1}}
\usepackage{enumitem}
\usepackage{spverbatim}
\usepackage{fancyvrb}


\makeglossaries
\loadglsentries{0.2_Glossar}

% &as_qdr=y2 (find date)

\fancypagestyle{lscape}{
\fancyhf{} %Clears the header/footer
\fancyfoot{% Footer
\makebox[\textwidth][r]{% Right
  \rlap{\hspace{1cm}% Push out of margin by \footskip
    \smash{% Remove vertical height
      \raisebox{4.87in}{% Raise vertically
        \rotatebox{90}{\thepage}}}}}}% Rotate counter-clockwise
\renewcommand{\headrulewidth}{0pt}% No header rule
\renewcommand{\footrulewidth}{0pt}% No footer rule
}

\begin{document}
  \begin{titlepage}
    \vspace*{2mm}
    \begin{center}
        \Large
        \textbf{Hochschule Worms}\\
        \textbf{Fachbereich Informatik}\\
        \textbf{Studiengang Angewandte Informatik B.Sc.}\\
        \vspace{3cm}
        \textbf{TBD}\\
        \vspace{1cm}
        \large
        Bacherloarbeit xxx\\
        \vspace{3cm}
        \begin {table}[ht]
        \centering
            \begin{tabular}{c}
                Bruno Macedo da Silva  \\ 
                676839                \\
                inf3645@hs-worms.de   \\
                Bebelstraße 22 Z10    \\
                67549 Worms            \\
            \end{tabular}
        \end {table}
        \vspace{2cm}
        \large
        \vspace{1cm}
        \begin{table}[h]
            \centering
            \begin{tabular}{l l}
                \multirow{2}{*}{Betreuer} & Prof. Dr. Zdravko Bozakov \\
                Bearbeitungszeitraum:     & Sommersemester 2023 \\
                Abgabedatum:              & xx. xxx 2023 \\
                Sperrvermerk:             & Ja/Nein \\
            \end{tabular}
        \end{table}    
    \end{center}
    \normalsize
    \vfill
\end{titlepage}
  \tableofcontents
  %\newpage
  %\addcontentsline{toc}{section}{\listfigurename}
  %\listoffigures
  \clearpage
  \printglossary[title=Glossar,toctitle=Glossar]
  \clearpage
  \newpage
  \clearpage
  \printglossary[type=acronym,title=Abkürzungen,toctitle=Abkürzungen]
  \clearpage
  \section{Einleitung}

Der heutige Netzwerkverkehr ist fast tausendfach größer als vor 20 Jahre \citep{Roser_I}. Das Internet wird heutzutage für fast alle unsere alltägliche Tätigkeit verwendet: Soziale Netzwerke, Video und Audio-Streaming, Einkauf, behördliche Angelegenheit und viele andere. So viel Verkehr generiert eine unermessliche Menge von Daten, die alle mögliche Inhalte beinhalten, von unschuldigen Anfragen nach dem eigenen Kontostand bis zu der Ausführung von bösewichten Anfragen, um Systemen lahmzumachen. Um das erste von der zweiten zu unterscheiden verwenden viele Firmen das sogenannte \glsfirst*{SIEM}.

Das \glsfirst{NIST} definiert \acrshort{SIEM} als Anwendung, die dafür zuständig ist, Sicherheitsdaten von anderen Systemen zu sammeln und diese verständlich und lesbar als Information zu liefern. Mit diesem Ergebnis können Aktionen durchgeführt werden \citep{NIST_SIEM}. Die Bewertung dieser Daten spielt eine wesentliche Rolle bei solchen Anwendungen, da es entscheidend ist, ob es um eine oder viele normale Anfrage oder um einen \gls{Cyberangriff} geht.

In diesem Projekt wollen wir über eine existierende \gls{Open Source} \gls{SIEM}-Anwendung recherchieren und ihre Extrahierung und Bewertung von Daten analysieren. Am Ende wollen wir uns für eine der gefundenen Lösungen entscheiden, sodass spezifische Logdateien der Hochschule Worms bewertet und bearbeitet werden können.

Diese Arbeit wird in folgende Teile geteilt:

% OSSIN: https://sourceforge.net/projects/os-sim/
% Preludes: https://www.prelude-siem.org/projects/prelude/wiki/ManualUser
% ELK Stack

% Grafana / Promtail: https://grafana.com/products/enterprise/
%https://grafana.com/logs/% 

%https://www.ossec.net/         https://github.com/ossec/ossec-rules

% was machen sie konkrent? / Vergleich zwischen OpenSource/Proprietary/

\begin{itemize}[noitemsep]
   \item Beschreibung von existierenden \glspl{SIEM} und Vergleich zwischen privaten Anbieter und eine \gls{Open Source} Lösungen (Alienvault OSSIN, OpenSearch, MozDef, Wazuh, Preludes)
   \item Analyse der Funktionalität einer \gls{Open Source} \gls{SIEM}
   \item Definition von zwei spezifische \glsplural{Cyberangriff}
   \item Empfang und Bearbeitung der Daten von den vorher beschriebenen Angriffe
   \item Entwicklung einer Regel für die Erkennung eines \gls{Cyberangriff}
   \item Analyse und Bewertung der Arbeit
\end{itemize}

Das folgende Diagramm stellt den Aufbau dieser Arbeit dar:

% Diagram anpassen mit korrenkten Zielen

\begin{figure}[H]
   \centering
   \includegraphics[width=0.8\textwidth]{assets/1_p1.jpg}
   \caption{Aufbau dieser wissenschaftlichen Recherche \gls{SIEM} \\Quelle: Eigene Darstellung }
   \centering
\end{figure}

\subsection{Problemstellung}
Während der Entwicklung dieser Arbeit wollen wir uns mit folgenden Fragen beschäftigen:

% Regeln anhand mittre, automatisieren


\begin{itemize}
   \item Welche Information-Muster muss von dem \gls{SIEM} extrahiert werden, um Angriff\_1 und Angriff\_2 zu erkennen?
   \item Wie sollen aussagekräftige Use-Cases / Regel sein, um Angriff\_2 und Angriff\_2 richtig zu erkennen zu erkennen?
   \item Wie sollen Server-Logs aussehen, damit sie von \glsplural{SIEM} bearbeitet werden können.
\end{itemize}

\textcolor{red}{\textbf{Note: Für Angriffe habe ich an DoS und Brute-Force (Password Spraying/Dictionary) gedacht.}}

\textcolor{red}{\textbf{Note 2: Punkt 3 wäre eher theoretisch, um zu recherchieren, was es schon gibt und was schon darüber geschrieben wurde.}}

\subsection{Vorgehensweise}
Um diese oben genannten Ziele zu erreichen, verwenden wir folgenden Methode:

\begin{itemize}[noitemsep]
   \item Recherche in der Fachliteratur über \glsplural{SIEM} Lösungen
   \item Vergleich zwischen verschiedenen \gls{Open Source} \gls{SIEM} Tools
   \item Installation von virtuellen Maschinen zur Nutzung von der ausgewählten \gls{SIEM} oder  
   \item Nutzung von Container zur Installation von \gls{SIEM}
   \item Importieren von Logdateien in der ausgewählten \gls{SIEM} Lösung
   \item Bewertung der ausgegebenen Daten
\end{itemize}


  \section{Problemstellung}

\begin{itemize}

   \item xxxx
   \item xxxx
   \item xxxx
   \item xxxx
   \item xxxx
\end{itemize}









  \section{Zielsetzung}

\begin{itemize}

   \item xxxx
   \item xxxx
   \item xxxx
   \item xxxx
   \item xxxx
\end{itemize}









  \section{Vorgehensweise}

Um diese obengenannten Ziele zu erreichen, verwenden wir folgenden Methode:

\begin{itemize}[noitemsep]
   \item aaaaaaaaaaa
\end{itemize}









  \section{Definition von Cloud-Computing}

Definition aus der Literatur.

\subsection{Art von Cloud-Umgebung}

Name der wichtigsten Anbieter.

Art:
\begin{itemize}
   \item IaaS
   \item PaaS
   \item IaaS
   \item SaaS
   \item DaaS
\end{itemize}




  \section{Provisionierung und Deployment von Cloud-Infrastruktur}

Vorgehensweise und Beispiele aus der Literatur.





  \section{Cloud Umgebung dieser Arbeit}

Erläuterung über die Verfügbare Umgebung für diese Arbeit.
Konkrete Maßnahmen für die Provisionierung und Deployment.





  \section{Fazit}

Zuammenfassung der Zielen und der Ergebnissen.


  \nocite{*}
  \begingroup
    \setlength{\bibsep}{4pt}
    \setstretch{1}
    \bibliography{9_Referenzen}
    \endgroup
  \bibliographystyle{apalike}
 % \setcitestyle{authoryear, open={\(},close={\)}
  %\newpage
    
\end{document}


% &as_qdr=y2