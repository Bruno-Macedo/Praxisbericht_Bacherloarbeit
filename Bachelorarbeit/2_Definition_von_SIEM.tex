\section{Definition von SIEM}

\gls{SIEM} ist das Ergebnis von der Kombination zwischen \glsfirst{SEM} und \glsfirst{SIM} \citep{Dorigo_SIEM}. Das erste bezieht sich auf der Identifizierung, Bewertung, Beobachtung und Bericht von Sicherheitsvorfällen mithilfe von verschiedenen Logdateien \citep{techopedia_SEM}. Das zweite ist ein Software, der bei der automatischen Sammlung von Loginformationen aus vielen Quellen, wie Firewall und Servers, untersützt \cite{techopedia_SIM}. 

In dem Universum von \glsfirst{SOC} mischen sich verschiedenen Begriffe, die manchmal zur Verwirrung führen, weil sie ähnliche Bedeutung und Verantwortung haben. \glsfirst{IDS}, \glsfirst{IPS} und \glsfirst{SIEM} werden von \textit{nonnative users} und sogar von Spezialisten oft verwechselt, da ihre Aufgabe mehr Zusammenhang als Unterschied haben. Um Umfang dieser Arbeit wegen der zeitlichen Einschränkungen zu verringen, fassen wir kurz die Unterschiede zwischen ihnen zusammen und legen wir unsere Grenze auf den \glspl{SIEM} Lösungen fest.

\glsfirst{IDS} sind Software oder Hardware die \glsplural{Cyberangriff} identifzieren und berichten. Sie haben eine passive Rolle, weil sie die \glsplural{Cyberangriff} weder stoppen noch verhindern können. \glsfirst{IPS} seinerseits haben eine aktive Haltung gegenüber \glsplural{Cyberangriff}, die können automatisch behandeln, indem sie Blocking-Mechanismus einschalten, um den Angriff zu stoppen \citep{Wendzel_IS}. Wie \gls{IDS}, kann der \gls{IPS} auch Logdateien generieren, die von einer \gls{SIEM} Lösung gesammelt werden kann. 

Die beiden ersten können innerhalb eines Unternehmen coexistieren, müssen aber nicht. Die Datenquellen von \glsplural{SIEM} können, unter anderen, von diesen beiden Tools entstehen. Die folgenden Abbildung stellt didaktisch, wie sich \glsplural{SIEM} in diesem Landschaft integrieren lassen: 

\begin{figure}[H]
   \centering
   \includegraphics[width=0.8\textwidth]{assets/2_p1.png}
   \caption{Allgemeine Informationsfluss von \gls{SIEM} \\Quelle: \citep{Mohanan_What} }
   \centering
\end{figure}

Aus dem Bild können wir feststellen, dass \glsplural{SIEM} für die Zentralisierung von Sicherheitsdaten zuständig ist. Diese werden dann bearbeitet und in einem oder mehreren Berichten dargestellt, damit das \gls{SOC}-Team schnellere und effektive Entscheidungen treffen können. Der Informationsfluss einer \gls{SIEM} Lösung können wirder in der folgenden Abbildung darstellen:

\begin{figure}[H]
   \centering
   \includegraphics[width=0.8\textwidth]{assets/2_p2.png}
   \caption{Allgemeine Informationsfluss von \gls{SIEM} \\Quelle: \citep{Granadillo_SIEM} }
   \centering
\end{figure}

\gls{SIEM} ist aber viel mehr als eine Sammlung von Logdateien. Das Ziel dieser Software ist die automatische Analyse zu ermöglichen, indem Daten kombiniert und bewertet werden können. In vielen Bereiche, wie Finanzen (\glsfirst{PCDISS}), Gesundheitswesen (\glsfirst{HIPAA}), sind \glsplural{SIEM} gesetzliche Verpflichtung \citep{Jog_SIEM}. In Deutschland verplichtet das \gls{IT-Sicherheitsgesetz 2.0} Organisationen mit kritischen Infrastrukturen die Anwendungen von solche Lösungen, um Störungen der \glsfirst{CIA} zu verhindern \citep{BSI_ITSG}.

\subsection{Existierende SIEM Lösungen}

Die existierenden \glsplural{SIEM} Lösungen können in zwei Kategorie getrennt werden: \textit{\gls{Proprietary}} und \textit{\gls{Open Source}}. Zu der ersten ist Splunk von dem Unternehmen Splunk Technology die meist verwendete Software \citep{Kazarov_Splunk}. Da unser Fokus hier auf \textit{\gls{Open Source}} Lösungen liegt, diskutieren wir hier demnächst über folgende Software:

\textbf{\textcolor{red}{Wie konnte ich Grafana hier erwähnen? Grafane ist eher algemmein und nicht so zu Alert orientiert, habe ich hier gefunden: \href{https://www.metricfire.com/blog/grafana-vs-splunk/}{Splunk x Grafana} und hier \href{https://www.researchgate.net/publication/350730340_Implementation_of_Grafana_as_open_source_visualization_and_query_processing_platform_for_data_scientists_and_researchers}{What is Grafana}}  }

\begin{itemize}[noitemsep]
   \item Prelude
   \item AlienVault
   \item ELK Stack
\end{itemize}

\subsubsection{Prelude}
aaaaaaa

\subsubsection{AlienVault OSSIM}
AlienVault OSSIM ist eine im Jahr 2007 entwickelte \gls{Open Source} SIEM Lösung. Im Jahr 2018 wurde sie von der Firma AT\&T Communication gekauft \citep{CBN_AV}. In der Beschreibung des Anbieters steht, dass sie auch dabei untersützt, Daten zu sammeln, zu normalisieren und zu bewerten. Er behaupt auch, dass sein Tool in der Lage ist, Schwachstelle und Angriffe zu erkennen, Verhältnis zu beobachten und Datenzusammenhang durchzuführen \citep{ATT_AVO}.

Laut der Website Comparitech steht AlienVault in der 13ten Platz von den besten bewerteten \gls{SIEM} Lösungen. Die Seite beschreibt auch, dass einen \gls{IDS}, Verhaltensüberwachungssystem und einen Schwacstellescanner integriert sind . Die Anwendung ist auch mit der Platform \gls{OTX} verbunden, diese ermöglicht die Teilung von Informationen über Schwachstelle. Comparitech highlighted, dass die Anwendung wegeren ihre nidriegen Kosten besser für kleine oder mittelständige Unternehmen geignet ist \citep{comparitech_SIEM}. 

Die Anwendung bietet konsitente Datenzusammenhang von Daten und vermeidet den Auftauch von \gls{falsch positiv}. AlienVault kommt auch mit vordefinierten Use-Cases, die dabei unterstützen gewöhnlichen Angriffsszenario zu erkennen. Die Installation, die Einstellung und die Integration mit anderen Tools ist auch benutzerfreundlich \citep{Gomes_AV}. Aus einer anderen wissenschaftlicher Quelle fanden wir heraus, dass es für viele  Quelle eine manuelle Normalisierung der Logdatein notwendig ist \cite{Nabil_AV}. Die Anwendunt hat aber einen zuverlässigen Berichtsmechanismus.


\subsubsection{ELK Stack}
cccccccccc

\subsubsection{Zusammenfassender Vergleich}
ddddddddd

\subsection{Existierende SIEM Lösungen}
eeeeeeeee

listen wir hier einige recherchierte Tools und präs

%https://community.microfocus.com/cyberres/b/sws-22/posts/sim-sem-and-siem-definitions-and-choosing-the-right-enterprise-solution





- Architektur %https://www.spiceworks.com/it-security/vulnerability-management/articles/what-is-siem/

- Anpassung %file:///C:/Users/bruno/Downloads/Security_Information_and_Event_Management_SIEM_Ana.pdf









