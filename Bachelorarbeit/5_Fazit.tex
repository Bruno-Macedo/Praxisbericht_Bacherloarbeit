\section{Fazit}

\subsection{Diskussion der Ergebnisse}
In dieser Arbeit versuchten wir, mithilfe von Grafana, Loki und Promtail eine \gls{SIEM}-Lösung zu emulieren, um Überwachungsmechanismen anhand von Logdateien zu erstellen. In der folgenden Tabelle zeigen wir die Rolle jedes verwendeten Tools bei der Erreichung unseres Ziels:

\begin{table}[h]
    \centering
    \setstretch{1.2}
    \begin{tabular}{|c|c|}
    \hline
    \textbf{Tool}    & \textbf{Funktionalität}         \\ \hline
    Promtail         & Datensammlung                   \\ \hline
    Loki             & Normalisierung und Vearbeitung  \\ \hline
    Grafana          & Berichts- und Grafikgenerierung \\ \hline
    Grafana:Alerting & Generierung von Warnmeldungen   \\ \hline
    \end{tabular}
    \caption{Verwendete Tools und ihre Hauptfunktionalitäten} 
    {Verwendete Tools und ihre Hauptfunktionalitäten}
    \label{tab:VerewendeteTools}
\end{table}

Wir stellen fest, dass die verwendeten Tools eine kosteneffektive Möglichkeit bieten, ein Überwachungssystem zu implementieren. Die Methoden zur Erkennung von Angriffen lassen sich anhand der \glsfirst{ttp} der \gls{mitre} Matrix definieren. Nach der Auswahl eines Angriffs erstellen wir Regelsätze mit der Abfragesprache \gls{logql} in Loki, um Muster zu identifizieren, die auf den ausgewählten Angriff hindeuten. Diese Regelsätze werden dann verwendet, um Warnmeldungen über den Angriff zu generieren und zu versenden.

Zu unsere initialen Ziele:

{\setstretch{1.5}
\begin{itemize}[noitemsep]
   \item Wie können wir ein Log-Analyse-Tool konfigurieren, dass es vordefinierte Angriffe nach der \gls{mitre} Matrix automatisch erkennen kann? 
   \item Wie können wir allgemeine Regelsätze definieren, sodass wir sie später für die verschiedene \gls{ttp} der \gls{mitre} Matrix anpassen können?
\end{itemize}
}

können wir sagen, dass die \gls{mitre} Matrix umfangreiche Informationen anbietet, um präzise Regelsätze zu generieren. 

\subsection{Herausforderungen}
Zu unserem primären Ziel können wir sagen, dass die \gls{mitre}-Matrix umfangreiche Informationen bietet, um zielgerichtete Regelsätze zu generieren. Die Erstellung dieser Regelsätze kann jedoch eine der größten Herausforderungen bei der Implementierung darstellen, da die Verwendung der Abfragesprache \gls{logql} viel Zeit in Anspruch nehmen kann. Sobald diese Hürde jedoch überwunden ist, ist es möglich, präzise Regelsätze zu erstellen, um potenzielle Angriffe zu identifizieren. Die Lernkurve für den Aufbau der richtigen Regelsätze kann eine große Herausforderung darstellen, wie auch in unserem Fall.

Da Logdateien aus produktiven Umgebungen eine große Menge an Informationen enthalten, müssen die Regelsätze so definiert werden, dass sie die relevanten Informationen wie IP-Adresse, Portnummer, Zeitfenster und Zeitabstände zwischen Anfragen filtern und nach Angriffsmustern kategorisieren können.

Die zweite große Herausforderung bestand darin, die richtigen Einstellungen und Funktionen von Promtail, Loki und Grafana zu verwenden. Das Beherrschen dieser Elemente kann dazu beitragen, dass die Anwendungen reibungslos funktionieren und vertrauenswürdige Ergebnisse liefern.

Die korrekte Nutzung der Filteroptionen von Promtail, genannt \quotes{scrape\_configs}, ermöglicht die Extrahierung spezifischer Informationen und die Generierung präziser Labels. Das Verständnis über die vielfältigen Funktionalitäten von Grafana trägt dazu bei, dass die ausgegebenen Daten die notwendigen Informationen enthalten, um den Entscheidungsprozess zu erleichtern. Die richtigen Einstellungen gewährleisten eine fehlerfreie Nutzung der Anwendungen und ermöglichen ihre Skalierbarkeit. In diesem Fall können sowohl die offizielle Dokumentation als auch die offiziellen Forenbeiträge dazu beitragen, die Tools richtig zu konfigurieren.

Letztendlich sind \quotes{Labels} auch wichtige Elemente bei Grafana, Loki und Promtail. Die korrekte und präzise Indizierung spielt eine entscheidende Rolle für die Leistung der Anwendung. Die Verwendung vieler \quotes{Labels} erfordert eine hohe Rechenkapazität und kann auch zu fehlerhaften Ergebnissen führen. Die Rechenkapazität muss ebenfalls angepasst werden, um sicherzustellen, dass die steigende Anzahl von Anfragen (siehe Abbildung \ref{fig:Eskalation_Labels} auf Seite \pageref{fig:Eskalation_Labels}) nicht zu Abstürzen führt.

\subsection{Zukünftige Forschung}
Dieser Arbeit ermöglicht eine Weiterentwicklugn in verschiedenen Bereichen:
 
\begin{itemize}[noitemsep]
    \item \textbf{Abdeckung vielen möglichen \glsplural{Cyberangriff}n mit neuen Regelsätze und Dashboards}:
\end{itemize}

Mit der Nutzung der \glsfirst{ttp} der \gls{mitre} Matrix ist es möglich, Regelsätze in \gls{logql} für viele andere \glsplural{Cyberangriff} aufzubauen und dadurch Logdateien aus verschiedenen Systemen und Anwendungen zu verwenden. Die Verwendung von anderen Regelsätzen, die sich auf verschiedene Angriffe beziehen, kann umfassende Sicherheit für produktive Umgebungen bieten, indem mehr \glsplural{usecases} abgedeckt werden, um Angriffe zu erkennen. Zur Unterstützung bietet Grafana bereits kundenspezifische Dashboards an, die verwendet und an die jeweilige Situation angepasst werden können \citep{Grafana_dashboards}.



\begin{itemize}[noitemsep]
    \item \textbf{Beherschung der Tools: Promtail, Loki und Grafana}:
\end{itemize}

Grafana, Loki und Promtail bieten in ihrer Konfiguration verschiedenen Möglichkeiten, um Informationen von Logdateien zu extrahieren, zu filtern und zu analysieren. Eine tiefe Beherrschung von \quotes{scrape\_configs} von Promtail trägt dazu bei, Logdateien zu erkennen und wichtige Informtionen direkt zu filtern, ohne das weitere Abfrage notwendig ist, indem auch Leistung gespart wird. Eine Weiterarbeit mit der \gls{abfragesprache} \gls{logql} hilft dabei, präzise und effizieren Abfrage aufzubauen, um bessere Grafiken und/oder Warnmeldung zu genieren. Zusätzlich kann die vielfältigen Funktionalitäten von Grafana dabei unterstützen, zuverlässige Grafiken und Tabelle zu generieren, um nützliche Informationen aus dem Logdateien grafisch darzustellen. Die Beherrschung dieses Tool stellt auch eine mögliche und vielversprechende weitere Recherche dar. 


\begin{itemize}[noitemsep]
    \item \textbf{Umfangreiche Beobachtbarkeit mit den Tools der \textit{\gls{GrafanaSystem}}}:
\end{itemize}

Die Tools um den \textit{\gls{GrafanaSystem}} bieten viele Möglichkeiten, um eine deutliche und akkurate Beobachtbarkeit eines Systems durchzuführen. Wenn kombiniert, ermöglichen sie eine holistische Analyse von \gls{pillarobservability}. Die kombinierte Implementierung in einer produktiven Umgebung kann dazu beitragen, die Sicherheit eines Systems auch bei skalierbaren Umgebungen zu verbessern. Eine Recherche in dieser Richtung hat auch die Möglichkeit, positive Ergebnisse zu liefern.


\begin{itemize}[noitemsep]
    \item \textbf{Automatische Antworten auf mögliche \glsplural{Cyberangriff}n}:
\end{itemize}

Eine umfangreiche \gls{SIEM}-Lösung bietet laut \cite{Mohammed_NOC} die wichtigen Informatinen, um Angriffe zu erkennen. Die Sicherheitsanalyse stoppt jedoch nicht in der Erkennung, sondern verlangt Handlungen, um laufende Angriffe zu stoppen oder potenziellen zu verhindern. Die Entwicklung oder die Integration von existierenden Tools, um automatisch gegen \glsplural{Cyberangriff} zu handeln, stellen auch eine mögliche Perspektive für zukünftige Recherche dar.

\begin{itemize}[noitemsep]
    \item \textbf{Nutzung von \gls{KI}}:
\end{itemize}

Moderne Angriffe haben heutzutage einen dynamischen Aspekt, der sich an die Umgebung anpasst, insbesondere durch die fortschreitende Entwicklung von \glsfirst{KI} \citep{Guembe_AIHACKER}. \gls{KI} kann zur Automatisierung von Aufgaben oder zur effizienten Datenanalyse eingesetzt werden. Für die zukünftige Weiterentwicklung dieser Arbeit kann dieses Werkzeug eine große Unterstützung bieten, indem es performantere und effizientere Regelsätze vorschlägt, um die Log-Analyse effizienter und zuverlässiger zu gestalten. 

Diese Möglichkeiten zusammen oder getrennt könnten dazu beitragen, einen sicheren Netzwerkverkehr zu gewährleisten.