\section{Fazit}

In dieser Arbeit versuchten wir, mithilfe von Grafana, Loki und Promtail eine \gls{SIEM} Lösung zu emulieren, um Überwachungsmechanismen anhand von Logdateien zu erstellen. In der folgenden Tabelle zeigen wir die Rolle jedes vewendeten Tools, um unser Ziel zu erreichen

\begin{table}[h]
    \centering
    \setstretch{1.2}
    \begin{tabular}{|c|c|}
    \hline
    \textbf{Tool}    & \textbf{Funktionalität}         \\ \hline
    Promtail         & Datensammlung                   \\ \hline
    Loki             & Normalisierung und Vearbeitung  \\ \hline
    Grafana          & Berichts- und Grafikgenerierung \\ \hline
    Grafana:Alerting & Generierung von Warnmeldung     \\ \hline
    \end{tabular}
    \caption{Verwendete Tools und ihre Funktionalitäten für den Aufbau einer \gls{SIEM} ähnlichen Lösung}
    {Verwendete Tools und ihre Funktionalitäten für den Aufbau einer \gls{SIEM} ähnlichen Lösung}
    \label{tab:VerewendeteTools}
\end{table}

Wir stellen fest, dass die verwendeten Tools mit Einschränkungen eine kosteneffektive Möglichkeit bieten, ein Überwachungssystem zu implementieren. Die Methoden zur Erkennung von Angriffen lassen sich anhand der \gls{ttp} der \gls{mitre}-Matrix definieren. Nach der Auswahl des Angriffe erstellen wir Regelwerke mit der Abfragesprache \gls{logql} in Loki, um Muster zu identifizieren, die auf den ausgewählten Angriff hindeuten. Diese Regelwerke werden dann verwendet, um Warnmeldungen über den Angriff zu generieren und zu versenden.

Die Einschränkungen beziehen sich auf die verschiedenen Problemen in der neuer Version von Loki, die seit 2022 gemeldet wurden und für die noch keine Lösung existiert.

\subsection{Herausforderungen}
In dem richtigen Aufbau der Regelsätze birgt die größte Herausfordung dieser Implementierung. Für eine präzise Anwendung der Tools spielt die richtige Entwicklung der Regelsätzen zur Identifizierung potenzieller Angriffe eine wichtige Rolle. Da Logdateien aus produktiven Umgebungen eine große Menge an Informationen enthalten, müssen diese Regelsätzen so definiert werden, dass sie die eindeutigen Informationen wie IP-Adresse, Portnummer, Zeitfenster und Zeitabstände zwischen Anfragen filtern und nach Angriffsmustern kategorisieren können. Die Lernkurve für den Aufbau der richtigen Regelsätzen kann eine große Herausforderung darstellen, was in unserem Fall war. 

Die richtige und präzise Indexierung kann einen entscheidende Faktor für die Leistung der Anwendung. Die Nutzung von vielen \quotes{labels} verlangt intensive Rechenkapazität und kann auch zu fehlerhafte Ergebnisse führen, wie in unserem Fall. Die Rechenkapazität soll auch angepasst werden, damit die steigende Anzahl von Anfragen (Siehe Abbildung \ref{fig:Eskalation_Labels} auf der Seite \pageref{fig:Eskalation_Labels}) nicht zu Abstürzt führt.

\subsection{Zukünftige Recherche}
%\textbf{\textcolor{red}{was kann in diesem Projekt erweitert werden.}}
%\textbf{\textcolor{red}{Neue Integration.}}

Die von uns definierten Regeln haben statische Elemente wie die \quotes{Anzahl von Anfragen}, den \quotes{Zeitabstand zwischen Requests} und die \quotes{Anzahl von fehlgeschlagenen Anmeldeversuchen}. Die heutigen Angriffe haben jedoch auch einen dynamischen Aspekt, der sich an die Umgebung anpasst, insbesondere durch die starke Entwicklung von \glsfirst{KI} \citep{Guembe_AIHACKER}. 

\gls{KI} kann für die Automatisierung von Aufgaben oder für effiziente Datenanalyse verwendet werden. Für die Weiterentwicklung dieser Arbeit kann dieses Tool große Unterstützung leisten, indem performanter und effizienter Regelsätze vorschlägt, um die Loganalyse effizienter und zuverlässiger zu machen. All dies würde dabei helfen, einen sicheren Netzwerkverkehr zu gewährleisten.


%Während \gls{KI} einerseits für die Automatisierung von Aufgaben oder für effiziente Datenanalyse verwendet wird, könnte sie auch für Cyberkriminalität genutzt werden. \gls{KI} ist am Ende nur ein Werkzeug, dessen Nutzung von den Absichten ihrer Benutzer abhängt.




% % % Verschiedene Angriffstechniken lassen sich schneller und effizienter mit \gls{KI} durchführen. Die Nutzung von \gls{polyphomicMalware} ist ein Beispiel, wo weder Antivirus-Programme noch Log-Analyse-Tools einen normalen von einem abnormalen Ablauf unterscheiden können. Auch die Verkehrsanalyse kann durch \gls{KI} gefährdet sein, da Angriffe und normaler Verkehr ähnlich dargestellt werden können. Darüber hinaus kann \gls{KI} auch gegen Authentifizierungsverfahren eingesetzt werden, um beispielsweise Anmeldedaten schneller zu \gls{bruteforce} und/oder vorauszusehen \citep{Fritsch_AIcybersec}.

% \newpage
% Das folgende Diagramm zeigt, wo sich \gls{KI} bei Cyberangriffen anhand der\gls{CKC} integrieren lässt:


% \begin{figure}[H]
%     \centering
%     \includegraphics[width=0.9\textwidth]{assets//CKC_AI.jpg}
%     \caption[\gls{KI} in der \glsfirst{CKC}]
%     {\gls{KI} in der \glsfirst{CKC}\\Quelle: \citep{Guembe_AIDiagrammAngriff}}
%     \centering
%  \end{figure}
 

% Um sicherzustellen, dass unsere Lösungen sich an diese neue und dynamische Realität anpassen können, können zukünftige Regelsätze mithilfe von \gls{KI} erstellt werden. Nachdem die meisten möglichen Angriffsflächen identifiziert wurden, sollten die Regeln die viele mögliche Szenarien abdecken.

% Mit der rasanten Entwicklung von \gls{KI}, insbesondere während der Erstellung dieser Arbeit, können wir auch erwarten, dass sich sowohl Loki als auch Grafana bald mit verschiedenen \gls{opensource} \gls{plugin} integrieren lassen, die auch \gls{KI} unterstützen. Diese sollen dazu beitragen, die Loganalyse effizienter und zuverlässiger zu machen. All dies würde dabei helfen, einen sicheren Netzwerkverkehr zu gewährleisten.


