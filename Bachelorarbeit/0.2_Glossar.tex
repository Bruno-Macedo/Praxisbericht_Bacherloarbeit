\newglossaryentry{abfragesprache} {
    name={Abfragesprache },
    description={Die \textit{Query Language} funktioniert wie ein Filter für die Suche nach spezifischen Daten in einer Datenbank \citep{at_query}}}


% here goes to the description
\newglossaryentry{API2}{
    name=\glslink{API}{Application Programming Interface (\gls{API})},
    description={bezieht sich auf Code und Regeln, die die Kommunikation zwischen verschiedenen Anwendungen ermöglichen. In diesem Fall kann eine Anwendung eine Anfrage an eine andere Anwendung senden, um Daten zu holen oder zu senden \citep{ibm_api}}}

% here goes to the acronym
\newglossaryentry{API}{
    type=\acronymtype,
    name=API,
    first={ Application Programming Interface (API)\glsadd{API2}},
    see=[Glossary:]{\gls{API2}}, 
    description=\glslink{API2}{Application Programming Interface}}
% =====

\newglossaryentry{bruteforce} {
    name={Brute-Force Angriffe},
    description={systematische Versuche, Zugangsdaten oder andere sensible Daten zu erraten, indem verschiedene Buchstaben, Ziffern und Symbole kombiniert werde \citep{Sowmya_BruteForce}}}

\newglossaryentry{BSI}{
    type=\acronymtype,
    name=BSI,
    first={ - Bundesamt für Sicherheit in der Informationstechnik (BSI)\glsadd{BSI}},
    see=[Glossary:]{\gls{BSI}}, 
    description=\glslink{BSI}{Bundesamt für Sicherheit in der Informationstechnik}}

\newglossaryentry{container} {
    name={Container},
    plural={Containers},
    description={funktionieren ähnlich wie virtuelle Maschinen (VMs), jedoch sind Container Anwendungen mit den notwendigen Ressourcen, um eingepackte Anwendungen auszuführen. Container werden oft für einzelne verwendet und teilen Ressourcen wie den Kernel des Host-Betriebssystems. Jeder Container ist in einer isolierten Umgebung mit den notwendigen Ressourcen für den Betrieb der ausgewählten Anwendung. Docker ist eine der bekanntesten Plattformen zur Verwaltung von Containern \citep{Douglis_Container}}}

\newglossaryentry{cortex} {
    name={Cortex},
    description={ist eine Open-Source-Plattform zur Verwaltung und Weiterverarbeitung von Sicherheitsvorfällen. Es fungiert als Analyse-Engine, indem es Informationen sammelt und je nach Fall Antworten oder Aktionen durchführt. Cortex kann eigenständig oder in Kombination mit anderen Tools verwendet werden \citep{TheHive}}   
}

\newglossaryentry{Cyberangriff} {
    name={Cyberangriff},
    plural={Cyberangriffe},
    description={Angriffe über Cyberspace. Solche Angriffe zielen darauf ab, Unternehmen und ihre Infrastrukturen zu zerstören, zu lähmen, zu kontrollieren oder die Integrität ihrer Daten zu stehlen oder zu manipulieren \citep{NIST_CyberAttack}}}

% here goes to the description
\newglossaryentry{CIA2}{
    name=\glslink{CIA}{Confidentiality, Integrity and Availability (\gls{CIA})},
    description={beschreiben die drei wichtigsten Schutzziele der IT-Sicherheit: Vertraulichkeit, Integrität und Verfügbarkeit \citep{Wendzel_IS}}}

% here goes to the acronym
\newglossaryentry{CIA}{
    type=\acronymtype,
    name=CIA,
    first={Confidentiality, Integrity and Availability (CIA)\glsadd{CIA2}},
    see=[Glossary:]{\gls{CIA2}}, 
    description=\glslink{CIA2}{Confidentiality, Integrity and Availability}}
% =====

% here goes to the description
\newglossaryentry{ttp2}{
    name=\glslink{ttp}{Taktiken, Techniken, Prozeduren (\gls{ttp})},
    description={beschreiben in der Mittre ATT\&CK Matrix Verhalten, Methode un Mustern bei Cyberangriffe \citep{Maymi_ttp}}}

% here goes to the acronym
\newglossaryentry{ttp}{
    type=\acronymtype,
    name=TTP,
    first={Taktiken, Techniken, Prozeduren (TTP)\glsadd{ttp2}},
    see=[Glossary:]{\gls{ttp2}}, 
    description=\glslink{ttp2}{Taktiken, Techniken, Prozeduren}}
% =====

% here goes to the description
\newglossaryentry{CKC2}{
    name=\glslink{CKC}{Cyber Kill Chain (\gls{CKC})},
    description={auch \textit{Cyberattack Lifecycle} genannt, bezieht sich auf ein Sicherheitsmodell für die Identifizierung, Analyse und Unterbrechung von fortgeschrittenen Cyberangriffen. Dieses Modell hat sieben festgelegte Phasen: \textit{Reconnaissance}, \textit{Weaponization}, \textit{Delivery}, \textit{Exploitation}, \textit{Installation}, \textit{Command \& Control (C2)} und \textit{Actions on Objectives} } \citep{Martin_CyberKillChain}}

% here goes to the acronym
\newglossaryentry{CKC}{
    type=\acronymtype,
    name=CKC,
    first={Cyber Kill Chain (CKC)\glsadd{CKC2}},
    see=[Glossary:]{\gls{CKC2}}, 
    description=\glslink{CKC2}{Cyber Kill Chain}}
% =====

\newglossaryentry{Cybersicherheit} {
    name={Cybersicherheit},
    description={ - Diese Domäne umfasst Kenntnisse und Methoden für den Schutz, die Prävention und Wiederherstellung von elektronischen Kommunikationsmitteln und deren Inhalten. Dabei konzentriert sie sich auf deren Verfügbarkeit, Integrität, Authentizität, Vertraulichkeit und Nichtabstreitbarkeit. \citep{NIST_CyberAttack}}}

\newglossaryentry{Endpoint} {
    name={Endpoint},
    plural={Endpoints},
    description={bezieht sich auf Geräte oder Systeme, die mit dem Netzwerk verbunden sind. Diese können z.B. Handys, Servers, Computers, Sensoren sein \citep{Microsoft_Endpoint}}}

\newglossaryentry{hash} {
    name={Hashwerte},
    plural={Hashwerte},
    description={sind Zeichenfolgen, die durch Anwenden einer mathematischen Funktion (Hashfunktion) auf einen Text oder eine Datei erzeugt werden. Die Rückführung auf das ursprüngliche Objekt aus dem Hashwert sollte jedoch unmöglich sein \citep{Wendzel_IS}}}

% here goes to the description
\newglossaryentry{http2}{
    name=\glslink{http}{Hypertext Transfer Protocol (\gls{http})},
    description={ist die Grundlage des Internets. Dieses Protokoll definiert die Regeln für die Übertragung von Texten und Dateien im Internet. Das Protokoll verwendet acht Methoden, um die Kommunikation zwischen Clients und Servern herzustellen: \textit{GET, POST, HEAD, DELETE, CONNECT, OPTIONS, PUT} und \textit{TRACE} \citep{Chai_http} and \citep{tutorialspoint_http}}}

% here goes to the acronym
\newglossaryentry{http}{
    type=\acronymtype,
    name=HTTP,
    first={Hypertext Transfer Protocol (http)\glsadd{http2}},
    see=[Glossary:]{\gls{http2}}, 
    description=\glslink{http2}{Hypertext Transfer Protocol}}
% =====


\newglossaryentry{IDS}{
    type=\acronymtype,
    name=IDS,
    first={Intrusion Detection System (IDS)\glsadd{IDS}},
    plural={IDSs}
    see=[Glossary:]{\gls{IDS}}, 
    description=\glslink{IDS}{Intrusion Detection System}}

\newglossaryentry{IT-Sicherheitsgesetz 2.0} {
    name={Zweites Gesetz zur Erhöhung der Sicherheit informationstechnischer Systeme},
    description={ist das zweite Gesetz zur Erhöhung der Sicherheit informationstechnischer Systeme wurde im Jahr 2021 als verabschiedetes Bundesgesetz zur Erhöhung der Sicherheit von informationstechnischen Systemen besonders von den kritischen Infrastrukturen \citep{Harmes_ITSG}}}

\newglossaryentry{falsch positiv} {
    name={Falsch Positiv},
    description={ist eine Warnmeldung einer nicht vorhandenen Verwundbarkeit \citep{NIST_FP}}}

%===========
\newglossaryentry{GUI2}{
    name=\glslink{GUI}{grafische Benutzeroberfläche (\gls{GUI})},
    description={ - Es handelt sich dabei um eine visuelle Schnittstelle, die es dem Benutzer ermöglicht, mit Anwendungen mittels Symbolen und grafischen Elementen zu interagieren. Im Gegensatz dazu verwendet die textbasierte Benutzeroberfläche (CLI) Befehlszeilen und Texteingabe zur Steuerung von Anwendungen \citep{Fu_GUI}}}

% here goes to the acronym
\newglossaryentry{GUI}{
    type=\acronymtype,
    name=GUI,
    first={grafische Benutzeroberfläche (GUI)\glsadd{GUI2}},
    see=[Glossary:]{\gls{GUI2}}, 
    description=\glslink{GUI2}{grafische Benutzeroberfläche}}
% =====
\newglossaryentry{IPS}{
    type=\acronymtype,
    name=IPS,
    first={Intrusion Prevention System (IPS)\glsadd{IPS}},
    plural={IPSs}
    see=[Glossary:]{\gls{IPS}}, 
    description=\glslink{IPS}{Intrusion Prevention System}}

\newglossaryentry{FPO}{
    type=\acronymtype,
    name=FPO,
    first={Fachspezifische Prüfungsordnung (FPO)\glsadd{FPO}},
    see=[Glossary:]{\gls{FPO}}, 
    description=\glslink{FPO}{Fachspezifische Prüfungsordnung}}

\newglossaryentry{hydra} {
    name={Hydra},
    description={ist eine Open Source Tool für Brute-Force Angriffe \citep{kali_hydra}}}

%=====
\newglossaryentry{HIPAA2}{
    name=\glslink{HIPAA}{Health Insurance Portability and Accountability Act (\gls{HIPAA})},
    description={ist ein US-Bundesgesetz über den Schutz von sensitiven personenbezogenen Gesundheitsdaten \citep{HHS_PCI}}
}
\newglossaryentry{HIPAA}{
    type=\acronymtype,
    name=HIPAA,
    first={Health Insurance Portability and Accountability Act (HIPAA)\glsadd{HIPAA2}},
    see=[Glossary:]{\gls{HIPAA2}}, 
    description=\glslink{HIPAA2}{Health Insurance Portability and Accountability Act}}
% =====

\newglossaryentry{kali} {
    name={Kali},
    description={ist eine Open-Source-Linux-Distribution, die speziell auf den Einsatz von Sicherheitstools für Angriffe und Sicherheitstests ausgelegt ist \citep{kali_docs}}}

% here goes to the description
\newglossaryentry{KI2}{
    name=\glslink{KI}{Künstliche Intelligenz (\gls{KI})},
    description={bezeichnet die Fähigkeit, Maschinen menschenähnliche kognitive Fähigkeiten wie Verständnis, Entscheidungsfindung, Lernen und Problemlösung zu entwickeln \citep{Collins_KI}}}
% here goes to the acronym
\newglossaryentry{KI}{
    type=\acronymtype,
    name=KI,
    first={Künstliche Intelligenz (KI)\glsadd{KI2}},
    see=[Glossary:]{\gls{KI2}}, 
    description=\glslink{KI2}{Künstliche Intelligenz}}

\newglossaryentry{logql} {
    name={LogQL},
    description={ist eine für Grafana Loki entwickelte Abfragesprache. Sie wird verwendet, um Logdateien zu zusammenzustellen \citep{Grafana_logql}}}

% here goes to the description
\newglossaryentry{mfa2}{
    name=\glslink{mfa}{Multi-Faktor-Authentisierung (\gls{mfa})},
    description={bezeichnet ein Authentifizierungsverfahren, bei dem mindestens zwei unabhängige Komponenten zur Identitätsprüfung verwendet werden, um eine höhere Sicherheit zu gewährleisten. Zum Beispiel kann ein Benutzer aufgefordert werden, sich mit einem Passwort und einem Fingerabdruck oder einem Token und/oder einer Gesichtserkennung zu authentifizieren \citep{Ibrokhimov_MFA}}}

% here goes to the acronym
\newglossaryentry{mfa}{
    type=\acronymtype,
    name=MFA,
    first={Multi-Faktor-Authentisierung (MFA)\glsadd{mfa2}},
    see=[Glossary:]{\gls{mfa2}}, 
    description=\glslink{mfa2}{Multi-Faktor-Authentisierung}}
% =====


%=====
\newglossaryentry{ML2}{
    name=\glslink{ML}{Machine Learning (\gls{ML})},
    description={bezieht sich auf die Fähigkeit von Systemen, automatisch und menschenähnlich Probleme zu lösen und spezifische Aufgabe zu erledigen \citep{Janiesch_ML}}}

\newglossaryentry{ML}{
    type=\acronymtype,
    name=ML,
    first={Machine Learning (ML)\glsadd{ML2}},
    see=[Glossary:]{\gls{ML2}}, 
    description=\glslink{ML2}{Machine Learning}}
% =====

\newglossaryentry{mitre} {
    name={Mitre ATT\&CK},
    description={Abkürzung für \textit{Adversarial Tactics, Techniques and Common Knowledge}. Es bezieht sich auf eine weltweit zugängliche Wissensbasis mit detaillierter Beschreibung, Klassifizierung und Bekämpfung von verschiedenen Angriffstechniken \citep{Mitre_Definition}}}

\newglossaryentry{mimir} {
    name={Mimir},
    description={ ein in Grafana integriertes Tool, das ähnlich wie Grafana Loki funktioniert. Es ermgölicht skalierbare Dateispeicherung, Bearbeitung und Abfrage mit der Abfragesprache LogQl \citep{Grafana_mimir}}   
    }

%=====
\newglossaryentry{NIST2}{
    name=\glslink{NIST}{National Institute of Standards and Technology (\gls{NIST})},
    description={ist eine US-Behörde, die für die Regelungen, Vereinheitlichung und Weiterentwicklung von Standards im Bereich Informationstechnologie zuständig ist \citep{NIST_AboutNist}}}

\newglossaryentry{NIST}{
    type=\acronymtype,
    name=NIST,
    first={National Institute of Standards and Technology (NIST)\glsadd{NIST2}},
    see=[Glossary:]{\gls{NIST2}}, 
    description=\glslink{NIST2}{National Institute of Standards and Technology}}
% =====

\newglossaryentry{OTX}{
    type=\acronymtype,
    name=OTX,
    first={Open Threat Exchange(OTX)\glsadd{OTX}},
    see=[Glossary:]{\gls{OTX}}, 
    description=\glslink{OTX}{Open Threat Exchange}}

\newglossaryentry{LML}{
    type=\acronymtype,
    name=LML,
    first={Log Monitoring Lackey (LML)\glsadd{LML}},
    see=[Glossary:]{\gls{LML}}, 
    description=\glslink{LML}{Log Monitoring Lackey}}

\newglossaryentry{opensource} {
    name={Open Source},
    description={beschreibt Software, die folgende Voraussetzungen erfüllen: freie Verteilung, Kopierung, Modifizierung und Nutzung und keine Diskriminierung gegenüber Personen und/oder Gruppe \citep{OpenSource_Def}}}

\newglossaryentry{OSSIM}{
    type=\acronymtype,
    name=OSSIM,
    first={Open Source Security Information Management (OSSIM)\glsadd{OSSIM}},
    see=[Glossary:]{\gls{OSSIM}}, 
    description=\glslink{OSSIM}{Open Source Security Information Management}}

\newglossaryentry{spraying} {
    name={Password Spraying},
    description={ist ein Angriff gegen Anmeldedaten, indem mögliche Passwörter gegen verschiedenen viele Benutzernamen verwendet werden. Das Ziel dieses Angriffes ist eine Kontosperrung zu vermeiden, indem wenige Versuche pro Nutzer stattfindet \citep{Swathi_spraxy}}}

\newglossaryentry{stuffing} {
    name={Password Stuffing},
    description={ist ein Angriff gegen Passwörtern, indem bekannte Anmeldedaten von vorherigen Angriffen verwendet werden. Dieser Angriff basiert sich auf die Idee, dass Nutzer dasselbe Passwort für verschiedenen Systemen verwenden \citep{Nguyen_stuffing}}}
    
%=====
\newglossaryentry{PCIDSS2}{
    name=\glslink{PCDISS}{Payment Card Industry Data Security Standard (\gls{PCDISS})},
    description={sind Sicherheitsstandards, die von Unternehmen, die Kreditkarten akzeptieren, verarbeiten, speichern oder übertragen, eingehalten werden müssen \citep{Compliance_PCI}}}

\newglossaryentry{PCDISS}{
    type=\acronymtype,
    name=PCDI DSS,
    first={Payment Card Industry Data Security Standard (PCDI DSS)\glsadd{PCIDSS2}},
    see=[Glossary:]{\gls{PCIDSS2}}, 
    description=\glslink{PCIDSS2}{Payment Card Industry Data Security Standard}}
% =====

\newglossaryentry{polyphomicMalware} {
    name={Polymorphe Malware},
    description={sind Schadprogramme, deren Signatur sich ständig ändern, um nicht von Anti-Malware-Systemen erkannt zu werden \citep{Syuhada_Polymorphic}}   
    }

%=====
\newglossaryentry{NOC2}{
    name=\glslink{NOC}{Network Operations Center (\gls{NOC})},
    description={ist ein zentralisierter Bereich eines Unternehmens, der für die Überwachung und Verwaltung von Netzwerkaktivitäten verantwortlich ist. \citep{Mohammed_NOC}}}

\newglossaryentry{NOC}{
    type=\acronymtype,
    name=NOC,
    first={Network Operations Center (NOC)\glsadd{NOC2}},
    see=[Glossary:]{\gls{NOC2}}, 
    description=\glslink{NOC2}{Network Operations Center}}
% =====

%=====
\newglossaryentry{owasp2}{
    name=\glslink{owasp}{Open Web Application Security Project (\gls{owasp})},
    description={ist eine Non-Profit-Organisation, die sich auf die Gewährleistung der Sicherheit im Umgang mit Webanwendungen konzentriert. Die Organisation verbreitet Open-Source-Informationen über sichere Entwicklung, Dokumentation, Best Practices für sicheren Umgang im Internet und Bildung \citep{owasp}}
}

\newglossaryentry{owasp}{
    type=\acronymtype,
    name=owasp,
    first={Open Web Application Security Project (OWASP)\glsadd{owasp}},
    see=[Glossary:]{\gls{owasp2}}, 
    description=\glslink{owasp2}{Open Web Application Security Project}
}
%=====

\newglossaryentry{plugin} {
    name={Plugin},
    plural={Plugins},
    description={sind optionale Software-Komponenten, die weitere Funktionalitäten zu einer Anwendung hinzufügen \citep{ITS_Network}}}

\newglossaryentry{port} {
    name={Port},
    plural={Ports},
    description={ist eine numerische Identifikation eines Dienstes oder einer Verbindung. Es handelt sich um eine logische Adressierung, die zur Identifikation eines oder mehrerer Prozesse verwendet wird \citep{Tanenbaum_CN}}}

\newglossaryentry{prometheus} {
    name={Prometheus},
    description={ist ein Open-Source-Tool der Firma SoundCloud. Es dient der Überwachung und Erstellung von Warnmeldungen, die auf der Grundlage von vordefinierten Regeln konfiguriert werden \citep{Prometheus_doc}}}

\newglossaryentry{Proprietary} {
    name={Proprietär},
    plural={Proprietäre},
    description={bezieht sich auf Software, die einer Firma oder Person gehört. Für die Nutzung ist in der Regel der Kauf einer Lizenz erforderlich. In diesem Fall haben Kunden nur begrenzten oder keinen Zugriff auf den Quellcode \citep{Nexcess_P}}}

%=====
\newglossaryentry{regex2}{
    name=\glslink{RegExp}{Regular Expressions (\gls{RegExp})},
    description={sind Methode, Methode, um Muster in Zeichenketten zu beschreiben. Im Informatikbereich werden solche Ausdrücke verwendet, um spezifische Texte oder Einträge in Textdateien zu finden \citep{Qusef_Regex}}}

\newglossaryentry{RegExp}{
    type=\acronymtype,
    name=RegExp,
    first={Regular Expression (RegExp)\glsadd{regex2}},
    see=[Glossary:]{\gls{RegExp}}, 
    description=\glslink{RegExp}{Regular Expression}}
% =====

\newglossaryentry{rockyou} {
    name={Rockyou},
    description={ist eine Textdatei mit über 8 Milliarden Passwörtern im Klartext. Diese Datei stammt aus einem Angriff gegen Yahoo im Jahr 2009 und wird seitdem ständig aktualisier \citep{Mikalauskas_rockyou}}}

\newglossaryentry{SIEM}{
    type=\acronymtype,
    name=SIEM,
    first={Security Information and Event Management (SIEM)\glsadd{SIEM}},
    plural={SIEMs},
    see=[Glossary:]{\gls{SIEM}}, 
    description=\glslink{SIEM}{Security Information and Event Management}}

\newglossaryentry{SEM}{
    type=\acronymtype,
    name=SEM,
    first={Security Event Management (SEM)\glsadd{SEM}},
    plural={SEMs},
    see=[Glossary:]{\gls{SEM}}, 
    description=\glslink{SEM}{Security Event Management}}

\newglossaryentry{SIM}{
    type=\acronymtype,
    name=SIM,
    first={Security Information Management (SIM)\glsadd{SIM}},
    plural={SIMs}
    see=[Glossary:]{\gls{SIM}}, 
    description=\glslink{SIM}{Security Information Management}}

%=====
\newglossaryentry{SOC2}{
    name=\glslink{SOC}{Security Operations Center (\gls{SOC})},
    description={ist ein zentralisierter Bereich eines Unternehmens, der für die Überwachung, Identifizierung, Bewertung und Reaktion auf Sicherheitsvorfälle verantwortlich ist. \citep{Vielberth_SOC}}}

\newglossaryentry{SOC}{
    type=\acronymtype,
    name=SOC,
    first={Security Operations Center (SOC)\glsadd{SOC2}},
    see=[Glossary:]{\gls{SOC2}}, 
    description=\glslink{SOC2}{Security Operations Center}}
% =====

\newglossaryentry{Schwachstelle} {
    name={Schwachstelle},
    plural={Schwachstellen},
    description={Schwäche eines Systems \citep{Wendzel_IS}}}

%=====
\newglossaryentry{ssh2}{
    name=\glslink{ssh}{Secure Shell Protocol(\gls{ssh})},
    description={ist ein Netzwerkprotokoll, das eine verschlüsselte Verbindung zwischen Endpunkten bietet. SSH wird meistens für die Fernadministration von Computern verwendet. Dieses Protokoll ermöglicht die Erstellung einer sicheren Verbindung in einer unsicheren Umgebung \citep{Wendzel_IS}}}

\newglossaryentry{ssh}{
    type=\acronymtype,
    name=SSH,
    first={Secure Shell Protocol (SSH)\glsadd{ssh2}},
    see=[Glossary:]{\gls{ssh2}}, 
    description=\glslink{ssh2}{Secure Shell Protocol}}
% =====

\newglossaryentry{ubuntu} {
    name={Ubuntu},
    description={ist eine Linux-Distribution, die oft für Server, Clients und Internet of Things (IoT) verwendet wird \citep{Ubuntu_ubuntu}}}

\newglossaryentry{usecases} {
    name={Use Cases},
    plural={Uses Cases},
    description={sind narrative Beschreibungen der Interaktionen zwischen Systemen und Benutzern. Sie dienen der Anforderungserhebung für ein System \citep{Savic_UseCase}}}

\newglossaryentry{USM}{
    type=\acronymtype,
    name=USM,
    first={Unified Security Management (USM)\glsadd{USM}},
    see=[Glossary:]{\gls{USM}}, 
    description=\glslink{USM}{Unified Security Management}}

\newglossaryentry{Verwundbarkeit} {
    name={Verwundbarkeit},
    plural={Verwundbarkeiten},
    description={auch \textit{vulnerability} genannt. Es beschreibt eine von Angreifer ausnutzbare Schwachstelle \citep{Wendzel_IS}}}

%=====
\newglossaryentry{vm2}{
    name=\glslink{vm}{Virtuelle Maschine (\gls{vm})},
    description={ist eine Kopie der Hardware-Struktur mit einer eigenen Aufteilung von Ressourcen und einem eigenen Betriebssystem. Auf einer physischen Maschine, auch Host genannt, können mehrere solcher VMs ausgeführt werden. Sie emulieren ein echtes und unabhängiges System \citep{Tanenbaum_MBS}}}

\newglossaryentry{vm}{
    type=\acronymtype,
    name=VM,
    plural=VMs,
    first={virtuellen Maschine (VM)\glsadd{vm2}},
    see=[Glossary:]{\gls{vm2}}, 
    description=\glslink{vm2}{virtuelle Maschine}}
% =====

\newglossaryentry{webhook} {
    name={Webhook},
    plural={Webhooks},
    description={Webhooks funktionieren ähnlich wie APIs, ohne dass die Client-Seite nach Aktualisierungen fragen muss. Bei Webhooks sendet der Server die Aktualisierung, sobald sie verfügbar ist. Die Kommunikation findet in Echtzeit statt \citep{Tas_webhook}}}





% \newglossaryentry{Schwachstelle} {
%     name={Schwachstelle},
%     plural={Schwachstellen}
%     description={Schwäche eines Systems \citep{Wendzel_It-Sicherheit}}
% }
    