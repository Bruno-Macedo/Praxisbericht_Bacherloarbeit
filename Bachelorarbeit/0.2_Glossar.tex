
\newglossaryentry{Cyberangriff} {
    name={Cyberangriff},
    plural={Cyberangriffe},
    description={Angriffe, die über den Cyberspace stattfinden. Solche Angriffe zielen auf Unternehmen und deren Infrastrukturen, um sie zu zerstören, lähmen, kontrollieren oder die Integrität deren Daten zu stehlen oder zu dominieren \citep{NIST_CyberAttack}}
}


% here goes to the description
\newglossaryentry{CIA2}{
    name=\glslink{CIA}{Confidentiality, Integrity and Availability (\gls{CIA})},
    description={Beschreibt die drei wichtigsten Schutzziele der IT-Sicherheit: Vertraulichkeit, Integrität und Verfügbarkeit \citep{Wendzel_IS}}
}
% here goes to the acronym
\newglossaryentry{CIA}{
    type=\acronymtype,
    name=CIA,
    first={Confidentiality, Integrity and Availability (CIA)\glsadd{CIA2}},
    see=[Glossary:]{\gls{CIA2}}, 
    description=\glslink{CIA2}{Confidentiality, Integrity and Availability}
}
% =====


\newglossaryentry{Cybersicherheit} {
    name={Cybersicherheit},
    description={Diese Domäne umfasst Kenntnisse und Methoden für den Schutz, Prävention, Wiederherstellung von elektronischen Kommunikationsmittel und deren Inhalt. Es konzentriert sich auf ihrer Verfügbarkeit, Integrität, Authentizität, Vertraulichkeit und Verbindlichkeit \citep{NIST_CyberSecurity}}
}

\newglossaryentry{FPO}{
    type=\acronymtype,
    name=FPO,
    first={Fachspezifische Prüfungsordnung (FPO)\glsadd{FPO}},
    see=[Glossary:]{\gls{FPO}}, 
    description=\glslink{FPO}{Fachspezifische Prüfungsordnung}
}

% here goes to the description
\newglossaryentry{KI2}{
    name=\glslink{KI}{Künstliche Intelligenz (\gls{KI})},
    description={Der Konzept, dass Maschine kognitive menschliche Fähigkeit entwickeln können, wie Verständnis, Entscheitungstreffen, Lernen, Problemlösung und \citep{Collins_KI}}
}
% here goes to the acronym
\newglossaryentry{KI}{
    type=\acronymtype,
    name=KI,
    first={Künstliche Intelligenz (KI)\glsadd{KI2}},
    see=[Glossary:]{\gls{KI2}}, 
    description=\glslink{KI2}{Künstliche Intelligenz}
}
% =====


%=====
\newglossaryentry{NIST2}{
    name=\glslink{NIST}{National Institute of Standards and Technology (\gls{NIST})},
    description={US-Behörden, die für die Reglungen, Vereinheitlichung und Weiterentwicklung im Bereich Informationstechnologie zuständig ist \citep{Hochschule_Worms_FPO} }
}
\newglossaryentry{NIST}{
    type=\acronymtype,
    name=NIST,
    first={National Institute of Standards and Technology (NIST)\glsadd{NIST2}},
    see=[Glossary:]{\gls{NIST2}}, 
    description=\glslink{NIST2}{National Institute of Standards and Technology}
}
% =====

\newglossaryentry{Open Source} {
    name={Open Source},
    description={beschreibt Code, die jeder zugreifen, modifiziren und verbreiten können, ohne dafür Lizenzen bezahlen zu müssen \citep{OpenSource_Def}}
}


\newglossaryentry{port} {
    name={Port},
    plural={Ports},
    description={ist eine Zahl, die ein Dienst oder eine Verbindung identifiziert. Es geht hier um eine logische Adressierung zur Identifizierung eines oder mehrere Prozessen \citep{Tanembaum_CN}}
}

\newglossaryentry{SIEM}{
    type=\acronymtype,
    name=SIEM,
    first={Security Information and Event Management (SIEM)\glsadd{SIEM}},
    plural={SIEMs}
    see=[Glossary:]{\gls{SIEM}}, 
    description=\glslink{SIEM}{Security Information and Event Management}
}

\newglossaryentry{Schwachstelle} {
    name={Schwachstelle},
    plural={Schwachstellen},
    description={Schwäche eines Systems \citep{Wendzel_IS}}
}

\newglossaryentry{Verwundbarkeit} {
    name={Verwundbarkeit},
    plural={Verwundbarkeiten},
    description={Auch \textit{vulnerability} genannt. Es beschreibt eine von Angreifer ausnutzbare Schwachstelle \citep{Wendzel_IS}}
}










% \newglossaryentry{Schwachstelle} {
%     name={Schwachstelle},
%     plural={Schwachstellen}
%     description={Schwäche eines Systems \citep{Wendzel_It-Sicherheit}}
% }
    