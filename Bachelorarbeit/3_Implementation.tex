\section{Implementierung}
In diesem Kapitel geht es um die Implementierung und den Aufbau von Grafana, um \gls{Cyberangriff} mithilfe der \gls{mitre} Matrix zu erkennen. Das Labor wird mit einem \gls{container} und \glsfirst{vm} aufgebaut, wie im Diagramm in der folgenden Abbildung dargestellt.

\begin{figure}[H]
   \centering
   \includegraphics[width=1\textwidth]{assets/Arbeitslabor.jpg}
   \caption[Aufbau unseres Arbeitslabors]
   {Aufbau unseres Arbeitslabors \\Quelle: Eigene Quelle}
   \centering
\end{figure}

Unser Aufbau verfolg folgende Ziele: die Aufnahme und Anpassung von Logdateien für Grafana, die Mustererkennung für ausgewählte \glsplural{Cyberangriff} und schließlich die Erstellung von Warnmeldungen für die Endnutzer, damit sie geeignete Sicherheitsmaßnahmen ergreifen können.

\newpage
Der gezielte Ablauf unserer Arbeit ist in dem folgenden Diagramm dargestellt:

\begin{figure}[H]
   \centering
   \includegraphics[width=1\textwidth]{assets/Ablauf_grafana2.jpg}
   \caption[Erwarteter Ablauf der Sammlung der Logdateien bis zur Warnmeldung]
   {Erwarteter Ablauf der Sammlung der Logdateien bis zur Warnmeldung \\ Quelle: Eigene Quelle und \citep{Grafana_loki}}
   \centering
\end{figure}

\subsection{Angriffserkennung anhand der Mitre ATT\&CK Matrix}
Es gibt verschiedene Methoden und Frameworks, die von \gls{SOC}-Teams verwendet wird, um \glsplural{Cyberangriff} zu vermeiden, zu erkennen und zu unterbrechen.  Da sich die Richtlinien und Schwerpunkte dieser Frameworks und Methoden unterscheiden können und somit unterschiedliche Anforderungen an den Aufbau unserer Struktur stellen könnten, entschieden wir uns für die \gls{mitre} Matrix, insbesondere da dieses Framework auch in Splunk integriert ist. 

Die \gls{mitre} Matrix hat folgenden Zweck \citep{Mitre_Started}:

{\setstretch{1.5}
\begin{itemize}[noitemsep]
   \item Erkennung und Analyse von Angriffstechnik
   \item	strukturierte Datensammlung über Bedrohungen
   \item	Emulieren von \glsplural{Cyberangriff} für die Anwendung an Angriffsübungen
   \item	Systemhärtung und Verbesserung der Verteidigungsmaßnahmen
\end{itemize}
}

Die Matrix ermöglichen Unternehmen und \gls{SOC}-Teams umfassende Möglichkeiten, um ihre Ressourcen zu schützen und ihr Fachwissen im Bereich der \gls{Cybersicherheit} zu erweitern \citep{Hazel_howtousemitre}. In dieser Arbeit konzentrieren wir uns auf die Entwicklung und Implementierung einer Methode zur automatischen Erkennung und Analyse von Angriffstechniken in Grafana.

Die \gls{mitre} Matrix ist auf \glsfirst{ttp} basiert. Angriffe, Gegenmaßnahmen und Erkennung werden nach \gls{ttp} definiert. Die Matrix besteht aus 14 Taktiken, zu denen jeweils Techniken gehören, die wiederum in Sub-Techniken unterteilt sind. Jede Sub-Technik wird mit Beispielen, Härtungsmaßnahmen und Erkennungsregeln beschrieben. Die nächste Abbildung, \ref{fig:ttp}, zeigt, wie die \gls{ttp} aufgebaut werden:

\begin{figure}[H]
   \centering
   \includegraphics[width=0.8\textwidth]{assets/Mitre_structure.drawio.png}
   \caption[Struktur der \gls{mitre} Matrix]
   {Struktur der \gls{mitre} Matrix \\Quelle: Eigene Quelle und \citep{Mitre_Started}}
   \label{fig:ttp}
   \centering
\end{figure}

% {\setstretch{1}
% Die 14 Taktiks sind folgende:
% \begin{itemize}[noitemsep]
%    \item Informationssammlung für zukünftige Angriffe 
%    \item	Entwicklung von Ressource von Angreifer
%    \item Erster Zugang zum Opfersysteme 
%    \item Ausführung von bösartigen Coden
%    \item Beharrlichkeit von System
%    \item	Privilegienausweitung
%    \item Vermeidung von Verteidigungssysteme
%    \item \textbf{Zugang zu Anmeldedaten}
%    \item Umgebungserkennung
%    \item Seitliche Bewegung zu anderem Systemen innerhalb des Angriffsziels
%    \item interne Informationssammlung
%    \item Steuerung und Kontrolle (C2 - Command and Control im Original)
%    \item Datenextrahierung 
%    \item	Auswirkung auf die Integrität
% \end{itemize}
% }

\newpage
\subsection{Auswahl des Angriffes}
In dieser Arbeit beschäftigen wir uns mit der Taktik \quotes{Zugang zu Anmeldedaten} und deren Technik \gls{bruteforce}. Diese Technik ist in vier Untertechniken aufteilt:

In dieser Arbeit beschäftigen wir uns mit der Taktik \quotes{Zugang zu Anmeldedaten} und ihrer Technik \quotes{\gls{bruteforce}}. Diese Technik ist in vier Untertechniken unterteilt:

{\setstretch{1}
\begin{itemize}[noitemsep]
   \item \gls{bruteforce}
   \item	Entschlüsselung von \glsplural{hash}
   \item \textit{\gls{stuffing}}
   \item \textit{\gls{spraying}}
\end{itemize}
}

Da unser Ziel hier ist, Grafana zu verwenden, um Angriffe zu erkennen, haben wir uns für einen einfachen und reproduzierbaren Angriff entschieden, der wenige Ressourcen erfordert. In diesem Fall kann ein \gls{bruteforce} mit zwei \glsplural{vm} problemlos durchgeführt werden. Für diesen Angriff verwenden wir die Sub-Technik "Erraten von Anmeldedaten und \textit{\gls{stuffing}}, da sie ähnliche Erkennungsmethoden aufweisen. Da unser Fokus bei dieser wissenschaftlichen Arbeit auf der Angriffserkennung schließen andere Maßnahmen  wir hierbei aus.

Die nächste Abbildung zeigt den Umfang unseres Implementationsversuchs mithifel von \gls{mitre}:
\begin{figure}[H]
   \centering
   \includegraphics[width=0.7\textwidth]{assets/T1110.drawio.png}
   \caption[\glsfirst{ttp} für unseren Angriff]
   {\glsfirst{ttp} für unseren Angriff \\Quelle: Eigene Quelle und \citep{Mitre_t1110}}
   \centering
\end{figure}

\newpage
\subsection{Installation und Generierung von Logdateien}
In diesem Abschnitt konzentrieren wir uns auf die folgenden Punkte:

\begin{enumerate}[noitemsep]
   \item Einrichtung von \glsplural{vm} für das Opfersystem und den Angreifer
   \item Simulation des Angriffe zur Erzeugung von Logdateien
   \item Installation und Konfiguration von Grafana Loki und Promtail mit \gls{container}
   \item Weiterleitung der Logdateien an Grafana
\end{enumerate}

Die Installation und Verwendung können entweder über eine \glsfirst{GUI} des Betriebssystems oder über die Kommandozeile durchgeführt werden. In dieser Arbeit verwenden wir die Kommandozeile.

\subsubsection{Einrichtung der \glsplural{vm} für Opfersystem und Angreifen}
Die beiden \glsplural{vm} sind eine \quotes{\gls{kali} \glsfirst{vm}} und \quotes{\gls{ubuntu} Server 22.04.2} mit standardmäßigen Einstellungen. Beide Maschinen wurden entsprechend ihrer jeweiligen Dokumentation installiert \citep{kali_vm} und \citep{Ubuntu_server}.

Für das Opfersystem haben wir uns für die Passwörter \quotes{qwertz} und \quotes{password} entschieden. Laut einer Umfrage gehören diese Passwörter zu den zehn am häufigsten verwendeten Passwörtern in Deutschland \citep{silicon_passwort}. Für die Durchführung des \gls{spraying} haben wir folgende Benutzername-Passwort Kombinationen erstellt:

{\setstretch{1.0}
\begin{lstlisting}[frame=single]
            Opfersystem 1          Opfersystem 2  
            admin:123456           bob:hallo
            user1:passwort         master:alice
            user2:abc123           hans:daniel
            user3:qwertyuiop       bruno:super123
\end{lstlisting}
}

\newpage
\subsubsection{Generierung von Logdateien mit der Angrifsssimulation}
Für den Angriff verwenden wir folgende Tools:

{\setstretch{1.0}
\begin{itemize}[noitemsep]
   \item	\glsfirst{ssh}
   \item \gls{hydra}
\end{itemize}
}

In diesem Szenario sendet \gls{hydra} gleichzeitig mehrere Authentifizierungsversuche an das Opfersystem, um eine \gls{ssh}-Verbindung herzustellen. Das Tool verwendet ein sogenanntes Wörterbuch mit verschiedenen Einträgen, die als Passwörter dienen. Für unseren Test benutzen wir die bekannte \gls{rockyou}-Wörterbuch.

Die folgende Abbildung, \ref{fig:stuffing}, zeigt, wie das \gls{stuffing} abläuft:

\begin{figure}[H]
   \centering
   \includegraphics[width=1\textwidth]{assets/Stuffing.jpg}
   \caption[Darstellung von \textit{\gls{stuffing}}]
   {Darstellung von \textit{\gls{stuffing}}\\Quelle: Eigene Quelle und \citep{Nguyen_stuffing}}
   \label{fig:stuffing}
   \centering
\end{figure}

In diesem Angriff versucht der Angreifer sich mit einem Konto anzumelden, indem er mit vielen Passwörter aus dem Wörterbuch probiert, bis eins richtige gefunden ist. Es können mehrere Anmeldungsversuche geschickt werden, bis eine von denen funktioniert.

\newpage
\gls{stuffing} wurde mit folgendem Kommando durchgeführt \citep{kali_hydra}:
%Verbatim
{\setstretch{1.0}
\begin{Verbatim}[frame=single]
hydra -l [Benutzername] -P rockyou.txt [Opfersystem] ssh -V -t 4

# Erklärung
-l: Spezifikation des Benutzernamens, den wir angreifen
-P: Auswahl der Datei mit bekannten Passwörtern
ssh: Auswahl der Anwendung, die wir angreifen
-V: Ausführliche Ausgabe über Versuche, Fehler und Erfolg
-t 4: Anzahl von gleichzeitigen Verbindungen
\end{Verbatim}
}

Die folgende Abbildung, \ref{fig:Stuffing_Opfer1}, zeigt einen Teil der Ausgabe von \gls{hydra} während der Ausführung von \gls{stuffing} gegen das Opfersystem1:
\begin{figure}[H]
   \centering
   \includegraphics[width=0.9\textwidth]{assets/stuffing_kali.png}
   \caption[Ausführung von \textit{\gls{stuffing}} gegen Opfersystem1]
   {Ausführung von \textit{\gls{stuffing}} gegen Opfersystem1\\Quelle: Eigene Quelle und \citep{Nguyen_stuffing}}
   \label{fig:Stuffing_Opfer1}
   \centering
\end{figure}

Und gegen Opfersystem2:
\begin{figure}[H]
   \centering
   \includegraphics[width=1\textwidth]{assets/stuffing_kali2.png}
   \caption[Ausführung von \textit{\gls{stuffing}} gegen Opfersystem2]
   {Ausführung von \textit{\gls{stuffing}} gegen Opfersystem2\\Quelle: Eigene Quelle und \citep{Nguyen_stuffing}}
   \centering
\end{figure}

Unser nächster Angriff, \gls{spraying}, sieht wie folgende aus:
\begin{figure}[H]
   \centering
   \includegraphics[width=1\textwidth]{assets/Spraying.jpg}
   \caption[Darstellung von \textit{\gls{spraying}}]
   {Darstellung von \textit{\gls{spraying}}\\Quelle: Eigene Quelle und \citep{Swathi_spraxy}}
   \label{fig:spraying}
   \centering
\end{figure}

Aus der Abbildung, \ref{fig:spraying}, sehen wir, dass bei \gls{spraying}, weniger Passwörter im Vergleich zum \gls{stuffing} verwendet wird, aber es werden mit mehreren Benutzername probiert. In diesem Fall will der Angreifer  Kontosperrungen vermeiden und Unauffälig gegenüber existierenden Sicherheitsmaßnamen bleiben.

Für diesen Angriff benutzen wir folgendes Kommando:
{\setstretch{1.0}
\begin{Verbatim}[frame=single]
hydra -L username2.txt -P passwoerter.txt [Opfersystem2] ssh -V -t 4

# Erklärung
-L: Auswahl der Datei mit gefunden Benutzernamen
\end{Verbatim}
}

\newpage
In diesem Fall gehen wir davon aus, dass der Angreifer einige oder alle Benutzernamen bereits kennt. Da bei diesem Angriff weniger Anmeldeversuche pro Nutzer durchgeführt werden, verwenden wir eine selbst erstellte Datei mit weniger Passwörtern als die \gls{rockyou}-Datei. Unsere Datei enthält die am häufigsten verwendeten Passwörter in Deutschland \citep{silicon_passwort}.

Die nächsten Abbildungen,\ref{fig:spraying_opfer1} und \ref{fig:spraying_opfer2}, zeigen die Ausführung von \gls{spraying}:
\begin{figure}[H]
   \centering
   \includegraphics[width=1\textwidth]{assets/Spraying_Kali.png}
   \caption[Ausführung \textit{\gls{spraying}} in Kali Linux gegen Opfersystem1]
   {Ausführung \textit{\gls{spraying}} in Kali Linux gegen Opfersystem1\\Quelle: Eigene Quelle}
   \label{fig:spraying_opfer1}
   \centering
\end{figure}

\begin{figure}[H]
   \centering
   \includegraphics[width=1\textwidth]{assets/Spraying_Kali2.png}
   \caption[Ausführung \textit{\gls{spraying}} in Kali Linux gegen Opfersystem2]
   {Ausführung \textit{\gls{spraying}} in Kali Linux gegen Opfersystem2\\Quelle: Eigene Quelle}
   \label{fig:spraying_opfer2}
   \centering
\end{figure}

\newpage
\subsubsection{Installation und Einrichtung von Grafana Loki und Promtail}
Die offizielle Dokumentation von Grafana war nicht immer eindeutig in Bezug auf die Ausführung, daher haben wir auch auf externe Quellen zurückgegriffen, um die Einstellungen an unsere Umgebung anzupassen \citep{Polinowski_PGL}. Unten befinden sich die von Grafana zur Verfügung gestellten Konfigurationsdateien und Installationsverfahren \citep{GrafanaLoki_run}:

{\setstretch{1.0}
\begin{lstlisting}[frame=single]
wget https://raw.githubusercontent.com/grafana/loki/v2.8.0/cmd/
loki/loki-local-config.yaml -O loki-config.yaml
(die Datei wurde angepasst)

wget https://raw.githubusercontent.com/grafana/loki/v2.8.0/
clients/cmd/promtail/promtail-docker-config.yaml
-O promtail-config.yaml (die Datei wurde angepasst)

docker-compose -f docker-compose.yaml up 
\end{lstlisting}
}

Im Anhang \ref{appendix:orgGrafana} befinden sich die originalen und die angepassten Dateien in Anhang \ref{appendix:AngepasstGrafana}.

Die obigen Kommandos haben folgende Bedeutungen:
\begin{enumerate}[noitemsep]
   \item Herunterladen der Konfigurationsdatei von Loki
   \item Herunterladen der Konfigurationsdatei von Promptail
   \item Ausführung von dem \glsplural{container}, indem beide Konfigurationsdateien in eine eingepackt und angepasst wurden und schließlich von der \gls{container}-Anwendung gelesen werden
\end{enumerate}

%Für spezifische Versionen oder weitere Einstellungen bietet die Dokumentation umfangreiche Möglichkeiten an \citep{GrafanaLoki_run}.

Für diesen ersten Test wurden die Logdateien des Opfersystems manuell auf den \gls{container} übertragen, da wir hier nur eine Instanz von Promtail verwedent haben.

\newpage
\newgeometry{right=30mm, left=30mm} 
\thispagestyle{lscape}
\begin{landscape}
   Nach der Ausführung des Kommandos ist die Anwendung benutzbar, wie in dem folgenden Screenshot:
    \begin{figure}[H]
       % \centering
        \centerline{\includegraphics[width=1.5\textwidth]{assets/Installation_Grafana.png}}
        %\includegraphics[width=1.2\textwidth]{assets/5.4.2_1_Abb.jpeg}
        \caption[Screenshot der Willkommensseite von Grafana Loki]
        {Screenshot der Willkommensseite von Grafana Loki\\Quelle: Eigene Quelle und \citep{Grafana_Logs}}
        \centering
    \end{figure} 
\end{landscape}
\restoregeometry  

\subsubsection{Weiterleitung der Logdateien zu Grafana}
Grafana Loki bietet interne und externe Möglichkeiten Logdateien zu empfangen. Die internen beziehen sich auf Grafana Tools, während die externen unabhängige Methode von Grafana benutzen:

\begin{enumerate}[noitemsep]
   \item \textbf{interne Methode}:
   \begin{enumerate}[noitemsep]
      \item Promtail 
      \item Grafana Agents
   \end{enumerate}
   \item \textbf{externe Methode}
   \begin{enumerate}[noitemsep]
      \item \glsfirst{API}
      \item \gls{opentelemetry} 
   \end{enumerate}
\end{enumerate}

In unserer Arbeit verwenden wir \textbf{Promtail}, der in einem \gls{container} läuft. Diese Instanz sendet die von uns ausgewählten Logdateien an Grafana und verarbeitet alle Dateien innerhalb eines sogenannten \quotes{jobs}. Promtail kann Logdatein nur zu Grafana Loki oder zu anderen Promtail-Instanz schicken \citep{Grafana_Promtail}. Die folgende Abbildung, \ref{fig: Promtail_Diagram}, zeigt die \glsplural{Endpoint} (links), wo Promtail installiert ist. Jeder Instanz von Promtail schickt Logdateien zu Loki (mitte), wo die Daten dann in Grafana (rechts) dargestellt werden.

%Bei verschiedenen wird jeder Typ einem eigenen \quotes{job} zugewiesen \citep{Grafana_CollectLogs}. Jeder \quotes{job} hat seine eigenen Regeln, um nach gefilterten Informationen zu suchen. 

\begin{figure}[H]
   \centering
   \includegraphics[width=0.6\textwidth]{assets/Promtail_Diagramm.png}
   \caption[Promtail in jeden \glsplural{Endpoint} kommuniziert sich mit Grafana Loki]
   {Promtail in jeden \glsplural{Endpoint} kommuniziert sich mit Grafana Loki\\Quelle: \citep{SpringCloud_Promtail}}
   \label{fig:Promtail_Diagramm}
   \centering
\end{figure}

In einer produktiven Umgebung ist die Installation von \textbf{Grafana Agents} auf jedem \gls{Endpoint} eine andere Lösung, um Grafana Loki mit Logdateien zu füllen. Während Promtail Logdatei nur zu Loki schickt, kann Grafana Agents Logdateien zu \gls{prometheus}, OpenTelemetry und zu Tools von \gls{GrafanaSystem}, wie \gls{mimir}, \gls{tempo}, \gls{phlare}, Loki und Grafana \citep{Grafana_Agents}. Die nächste Abbildung, \ref{fig:GrafAgents}, zeigt den Kommunikationfluss zwischen Grafana Agents und die integrieten Tools:

\begin{figure}[H]
   \centering
   \includegraphics[width=0.6\textwidth]{assets/GrafanaAgents.drawio.png}
   \caption[Kommunikation zwischen Grafana Agents, \gls{prometheus}, OpenTelemetry und \gls{GrafanaSystem}]
   {Kommunikation zwischen Grafana Agents, \gls{prometheus}, OpenTelemetry und \gls{GrafanaSystem}\\Quelle: \citep{Grafana_Agents}}
   \label{fig:GrafAgents}
   \centering
\end{figure}

Der Kommunikationsfluss bei Grafana Agents funktioniert ähnlich, wie bei Promtail. Die \glsplural{Endpoint} (links), wo die Agents installiert sind, schicken die Logdateien zu den kompatiblen Tools (mitte), die sich wiederum mit Grafana (rechts) kommunizieren.

Die Sendung des Inhalts der Logdateien findet auch mithilfe von Grafana Loki \gls{http} \textbf{\gls{API}} statt. In diesem Fall werden die Zeile der Logdateien und nicht der Datei zum \gls{Endpoint} von Loki mit \gls{http} POST-Anfrage geschickt. 

Grafana Loki bietet auch eine Integration mit dem Open-Source-Tool \gls{opentelemetry} an, um Logdateien zu empfangen \citep{Grafana_opentelemetry}. Die Integration mit Grafana Loki erfolgt über die Nutzung von \glsplural{API}. Der \textit{Collector} läuft in derselben Umgebung wie Grafana Loki, damit er die Logdateien empfangen und verarbeiten kann. Die \textit{Agents} laufen auf jedem Endpunkt und kommunizieren mit dem \textit{Collector}. Die folgende Abbildung stellt das Kommunikationsverfahren zwischen \gls{opentelemetry} und Grafana Liki:

\begin{figure}[H]
   \centering
   \includegraphics[width=0.8\textwidth]{assets/Grafana_OpenTelemtry.png}
   \caption[Datenfluss zwischen \gls{opentelemetry} und die Tools von \gls{GrafanaSystem}]
   {Datenfluss zwischen \gls{opentelemetry} und\gls{GrafanaSystem}\\Quelle: \citep{Grafana_WhatOpentelemetry}}
   \centering
\end{figure}

Auf der linken Seite haben wir die verschiedenen \glsplural{Endpoint}, auf denen jeweils ein \textit{Agent} läuft. In der Mitte ist der \textit{Collector}, der die Logdateien schließlich an die Tools von \gls{GrafanaSystem} weiterleitet.

\newpage
\subsection{Aufbau der Erkennungsregel für den ausgewählten Angriff}
Ein \gls{bruteforce} lässt sich durch eine hohe Anzahl der fehlgeschlagenen Anmeldeversuche erkennen \citep{Selvaganesh_SplunkBruteForce}. Wir betrachten eine Situation, in der keine Gegenmaßnahmen wie Kontosperre nach \textit{n} beliebigen Versuchen oder \gls{mfa}, implementiert sind. Das folgende Abbildung, \ref{fig:Aktivitaetsdiagramm_Anmeldung}, stellt Diagramm mit einem allgemeinen Ablauf eines Anmeldungsverfahrens dar:

\begin{figure}[H]
   \centering
   \includegraphics[width=0.8\textwidth]{assets/Anmeldeverfahren.drawio.png}
   \caption[Allgemeiner Ablauf eines Anmeldungsverfahrens]
   {Allgemeiner Ablauf eines Anmeldungsverfahrens \\Quelle: Eigene Quelle und \citep{Selvaganesh_SplunkBruteForce}}
   \label{fig:Aktivitaetsdiagramm_Anmeldung}
   \centering
\end{figure}

Grafana bietet ein Konfigurationsmuster für die Eingabe und Darstellung von \gls{ssh} Eventds an. In dieser Konfiguration sind bereits Regesätze für die 
Verarbeirtung der Log-Einträge in Loki und Quellcode für die Generierung von Grafik in Grafana. Diese Konfigurationsdatei ermöglicht eine umfassende Analyse dieser Daten \citep{VoidQuark_sshlogs}. Die gesentet Logdateien werden mithilfe der folgenden Elemente gelesen und verarbeitet:

% % \begin{table}[H]
% %    \includegraphics[width=\linewidth]{assets/tabelle_sshgrafana.png}
% %    \caption[Elementen eines Regelsätzes in Grafana Loki]
% %    {Elementen eines Regelsätzes in Grafana Loki \\Quelle: Eigene Quelle, \citep{VoidQuark_sshlogs} und \citep{Setter_Logfmt}}
% % \end{table}


\begin{table}[H]
   \setstretch{1.2}
   \begin{tabularx}{\textwidth}{|c|X|}
   \hline
   \multicolumn{1}{|c|}{\textbf{Element}} & \multicolumn{1}{|c|}{\textbf{Beschreibung}} \\
   \hline
      json & Lesbare Dateiformat, deren Daten nach dem Regel Schlüssel:Wert gespeichert sind \\
   \hline
      Muster & Lesen und Extraktion der Information der Logdateien \\
   \hline
      \glsfirst{RegExp} & Mustererkennung aus der Logdatei \\
   \hline
      Logfmt & Extraktion von Schlüssel:Wert Paar der Logdateien \\
   \hline
   \end{tabularx}
   \caption[Elementen eines Regelsätzes in Grafana Loki]
   {Elementen eines Regelsätzes in Grafana Loki \\Quelle: Eigene Quelle, \citep{VoidQuark_sshlogs} und \citep{Setter_Logfmt}}
\end{table}


%https://grafana.com/docs/loki/latest/fundamentals/labels/
%https://prometheus.io/docs/concepts/jobs_instances/

Für jedes Angriffszenario benutzen wir spezifische Regeln, die mit \gls{logql} aufgebaut sind. Die Filterung findet mithilfe von zweil Labels \quotes{Instance} und \quotes{Job} statt. In Promtail wird jeder \gls{Endpoint} als \quotes{Instance} bezeichnet. Eine oder mehrere \quotes{Instances} werden einem \quotes{Job} zugewiesen. \quotes{Jobs} beziehen sich auf die Bearbeitung der Logdateien nach dem spezifizieren Regeln, in unserem Fall, Überprüfung von \gls{ssh}-Logdateien. Diese Struktur stammt aus dem Tool \gls{prometheus}. Alle unsere \quotes{Instance} werden in einem \quotes{Job} eingepackt, wo sie nach den gleichen Regeln verarbeitet.
Das folgende Diagramm stellt die Beziehung zwischen dieser beiden Labels dar:

\begin{figure}[H]
   \centering
   \includegraphics[width=0.8\textwidth]{assets/Instance_Jobs.drawio.png}
   \caption[Beziehung zwischen \quotes{Instance} und \quotes{Job}]
   {Beziehung zwischen \quotes{Instance} und \quotes{Job} \\Quelle: Eigene Quelle und \citep{Prometheus_JobInstance}}
   \centering
\end{figure}

In dem nächsten Abschnitt beschreiben wir, wie diese Regel in \gls{logql} geschrieben werden.

\newpage
\subsubsection{Regelsätze in LogQL}
In diesem Abschnitt fassen wir zusammen, wie eine Abfrage in \gls{logql} für eine Logdatei mit \gls{ssh} Einträgen aussieht. Für ausführliche Informationen über den Aufbau der Abfrage verweisen wir die offizielle Dokumentation, auf die diese Erklärung basiert ist \citep{Grafana_logql}. Unsere Logdatei enthält unter anderem folgende Zeile:

{\setstretch{1.0}
\begin{Verbatim}[frame=single]
14 14:05:30 opfersystem2 sshd[1698]: Failed password for administrator 
from 10.0.2.15 port 58036 ssh2
\end{Verbatim}
}

Um fehlgeschlagene Anmeldeversuche zu erkennen, extrahieren wir folgende Felder aus den \gls{ssh}-Logdateien. Diese Information verwenden wir um gleiche Events zu erkennen und deren Anzhal festzustellen. 

{\setstretch{1.0}
\begin{Verbatim}[commandchars=\\\{\},frame=single]
14 14:05:30 opfersystem2 \textbf{\textcolor{red}{sshd[}}1698]: \textbf{\textcolor{red}{Failed}} password for \textbf{\textcolor{blue}{administrator}}
from \textbf{\textcolor{blue}{10.0.2.15}} port 58036 ssh2
\end{Verbatim}
}

Wir teilen die Abfrage unten mit, um ihre Bestandteile besser zu verstehen: 

\begin{table}[H]
   \setstretch{1.2}
   \begin{tabularx}{\textwidth}{|m{5cm}|X|}
   \hline
   \multicolumn{1}{|c|}{\textbf{Element}} & \multicolumn{1}{|c|}{\textbf{Beschreibung}} \\
   \hline
   \centering
   sum by(add)
   (rate(\{job=\verb|"|\textit{JOBNAME}\verb|"|
   instance=~\verb|"|\$instance\verb|"|\} 

   & Hiermit wird die Aufsummierung der Benutzernamen definiert, die wir mit \quotes{Patterns} in \gls{logql} definiert haben. \quotes{Patterns} ermöglichen die einfache Extrahierung von Informationen aus einer Zeile. Wir holen alle Log-Einträge, die sich auf den von uns definierten Job beziehen. Wir können auch nach spezifischen Endpoint filtern, indem wir das Schlüsselwort \quotes{instance} benutzen. \\
   \hline
   \centering
   | 
   & \quotes{|} funktioniert in \gls{logql} wie eine Pipeline für die Verkettung von mehreren Suchmustern. \\
   \hline
   \centering
         |= \lq \textbf{\textcolor{red}{sshd}}[\rq 
      \\ |= \lq: \textbf{\textcolor{red}{Failed}}\rq 
    & 
    Suche nach Zeilen mit den in den rot markierten Einträgen. \\
   \hline
   \end{tabularx}
\end{table}

\begin{table}[H]
   \setstretch{1.2}
   \begin{tabularx}{\textwidth}{|m{5cm}|X|}
   \hline
   \multicolumn{1}{|c|}{\textbf{Element}} & \multicolumn{1}{|c|}{\textbf{Beschreibung}} \\
   \hline
   \centering
      !\verb|~| \lq \textbf{\textcolor{red}{invalid user}}\rq \\
      !\verb|~| \lq \textbf{\textcolor{red}{Legitimer\_Nutzer}}\rq \\
      !\verb|~| \lq \textbf{\textcolor{red}{Legitime\_Adresse}}\rq 
   & Suche nach Zeilen \textbf{ohne} diese Einträge. Wir können beispielsweise Einträge ausschließen, die auf legitimen Nutzer oder IP-Adresse beziehen, um falsche Positive zu vermeiden \\
   \hline
   \centering
      | pattern \lq<\_> for \\
      <\textbf{\textcolor{blue}{Benutzername}}> from \\
      <\textbf{\textcolor{blue}{Quelladresse}}> port <\_>\rq \\
      \verb|[|\$ \verb|__|range\verb|]|\verb|))| 
      
   & Die Definition der Wörter \quotes{Benutzername}, \quotes{Quelladresse} und als \quotes{Patterns} dienen dazu, einen Benutzernamen und eine Quelle IP-Adresse aus der Logdatei zu extrahieren. Die Platzhalter \quotes{<\_>} sind unbenannte Elemente, die in diesem Fall auf die Einträge \quotes{password} und \gls{port} in der Zeile verweisen. \\
   \hline
   \end{tabularx}
   \caption[Elementen eines Regelsätzes in Grafana Loki]
   {Elementen eines Regelsätzes in Grafana Loki \\Quelle: Eigene Quelle, \citep{VoidQuark_sshlogs} und \citep{Setter_Logfmt}}
\end{table}


% \begin{table}[H]
%    \includegraphics[width=1\linewidth]{assets/tabelle_logql_1.png}
% \end{table}

% \begin{table}[H]
%    \includegraphics[width=1\linewidth]{assets/tabelle_logql_2.png}
%    \caption[Aufbau der Regelsätze in Grafana Loki für \gls{ssh} Logdateien]
%    {Aufbau der Regelsätze in Grafana Loki für \gls{ssh} Logdateien \\Quelle: Eigene Quelle, \citep{VoidQuark_sshlogs} und \citep{Grafana_logql}}
% \end{table}

Schließlich sieht der Regelsatz so aus:
\verb

{\setstretch{1.0}
\begin{Verbatim}[fontsize=\small, commandchars=\\\{\}, frame=single]
sum by(add) (rate({job="\textit{JOBNAME}", instance=~"$instance"} |= '\textbf{\textcolor{red}{sshd}}[' |= ': 
\textbf{\textcolor{red}{Failed}}' !~ '\textbf{\textcolor{red}{invalid user}}' !~ '\textbf{\textcolor{red}{Legitimer_Nutzer}}' !~ '\textbf{\textcolor{red}{Legitime_Adresse}}' |
pattern '<_> for <\textbf{\textcolor{blue}{Benutzername}}> from  <\textbf{\textcolor{blue}{Quelladresse}}> port <_>' [$__range]))
\end{Verbatim}
}

%sum by(add) (rate({job="JOBNAME", instance=~"$instance"} |= 'sshd[' |= ': Failed' !~ 'invalid user' !~ 'Legitimer_Nutzer' !~ 'Legitime_Adresse' | pattern '<_> for <Benutzername> from <Quelladresse> port <_>' [$__range]))

% \begin{table}[H]
%    \includegraphics[width=1\linewidth]{assets/tabelle_logql_2.png}
%    \caption[Aufbau der Regelsätze in Grafana Loki für \gls{ssh} Logdateien]
%    {Aufbau der Regelsätze in Grafana Loki für \gls{ssh} Logdateien \\Quelle: Eigene Quelle, \citep{VoidQuark_sshlogs} und \citep{Grafana_logql}}
% \end{table}


\textbf{\textcolor{red}{Das sollte verbessert werden}}
Eine Erkennungsregel hätte folgende Logik:
{\setstretch{1.0}
\begin{Verbatim}[frame=single]
   # Gefundene Werte in den Logdateien
   # Av = Anzahl fehlgeschlagener Anmeldungsversuche
   # Ia = Intervallzeit zwischen fehlgeschlagenen Anmeldungsversuchen

   # Festgelegte Werte für legitime und bösartige Verbindungen
   # Ga = Grenze zwischen legitimen und bösartigen Anmeldungsversuchen
   # Nt = Intervallzeit zwischen legitimen Anmeldungsversuchen

   wenn (Av >= Ga) und (Ia < Nt)
      Warnmeldung(Brutefoce)
   sonst
      weiterBeobachten()
\end{Verbatim}
}


\subsection{Hinzufügen der Regelsätze Grafana Loki}
\textcolor{red}{\textbf{TODO Citar regra API https://grafana.com/docs/loki/latest/api/}}

Die Regelsätze in Grafana Loki können sowohl manuell im Menü \quotes{Code} als auch über die \gls{GUI} im Menü \quotes{Builder} geschrieben werden. Letzteres bietet eine benutzerfreundlichere Umgebung, um die Regeln zu schreiben. Die folgenden Abbildungen zeigen diese beiden Optionen:

\begin{figure}[H]
   \centering
   \includegraphics[width=0.85\textwidth]{assets/manuellerCodeLoki.png}
   \caption[\quotes{Code} in Grafana Loki für manuelle die Eingabe des \gls{logql}-Codes]
   {\quotes{Code} in Grafana Loki für manuelle die Eingabe des \gls{logql}-Codes. \\ Quelle: \citep{VoidQuark_sshlogs}}
   \centering
\end{figure}

\begin{figure}[H]
   \centering
   \includegraphics[width=0.85\textwidth]{assets/klickibuntyGrafana.png}
   \caption[\quotes{Builder} in Grafana Loki für nutzerfreundlichere Eingabe des \gls{logql}-Codes.]
   {\quotes{Builder} in Grafana Loki für nutzerfreundlichere Eingabe des \gls{logql}-Codes. Quelle: \citep{VoidQuark_sshlogs}}
   \centering
\end{figure}

Beide Optionen bieten die Möglichkeit, eine Erklärung zur Abfrage anzuzeigen:
\begin{figure}[H]
   \centering
   \includegraphics[width=1\textwidth]{assets/erklaerungLoki.png}
   \caption[Ausführliche Information über die Abfrage]
   {Ausführliche Information über die Abfrage\\Quelle: \citep{Grafana_QueryEditor}}
   \centering
\end{figure}

\newpage
\newgeometry{right=30mm, left=30mm} 
\thispagestyle{lscape}
\begin{landscape}
   Nachdem die \gls{ssh}-Logdateien gelesen und bearbeiten wurden, bekommen wir von Grafana Loki folgende Zasammenfasung der Ergebnissen:
    \begin{figure}[H]
       % \centering
        \centerline{\includegraphics[width=1.7\textwidth]{assets/GrafanaLoki_sshDetailed.png}}
        %\includegraphics[width=1.2\textwidth]{assets/5.4.2_1_Abb.jpeg}
        \caption[Bearbeitung der \gls{ssh} Logdateien von Grafana Loki]
        {Bearbeitung der \gls{ssh} Logdateien von Grafana Loki\\Quelle: Eigene Quelle and \citep{VoidQuark_sshlogs}}
        \centering
    \end{figure} 
\end{landscape}
\restoregeometry

\newpage
\newgeometry{right=30mm, left=30mm} 
\thispagestyle{lscape}
\begin{landscape}
   Das nächste Bild gibt ausführliche Informationen der Logdateien:
    \begin{figure}[H]
       % \centering
        \centerline{\includegraphics[width=1.7\textwidth]{assets/GrafanaLoki_sshDetailed.png}}
        \caption[Ausführliche Darstellung der \gls{ssh} Logdateien von Grafana Loki]
        {Ausführliche Darstellung der \gls{ssh} Logdateien von Grafana Loki\\Quelle: Eigene Quelle and \citep{VoidQuark_sshlogs}}
        \centering
    \end{figure} 
\end{landscape}
\restoregeometry

% \thispagestyle{lscape}
% \begin{landscape}
%    Das nächste Bild gibt ausführliche Informationen der Logdateien:
%    \begin{center}
%       \begin{figure}[H]
%          \centering
%          \includegraphics[width=1.3\textwidth]{assets/GrafanaLoki_sshDetailed.png}.
%          \caption[Ausführliche Darstellung der \gls{ssh} Logdateien von Grafana Loki]
%          {Ausführliche Darstellung der \gls{ssh} Logdateien von Grafana Loki\\Quelle: Eigene Quelle and \citep{VoidQuark_sshlogs}}
%          \centering
%       \end{figure}
%    \end{center}
% \end{landscape}

\subsection{Einrichtung der Warnmeldungen in Grafana}
In den vorherigen Teilen dieser Arbeit haben wir uns damit auseinandergesetzt, Grafana so einzurichten, dass wir schließlich eine Lösung ähnlich einer \gls{SIEM} erhalten. Von unseren ursprünglichen Ziele haben wir bereits Folgendes erreicht:

{\setstretch{1}
\begin{enumerate}[noitemsep]
   \item	Sammlung der Logdateien von den \glsplural{Endpoint} mit Promtail
   \item Anpassung der Logdateien für die Weiterleitung an Grafana Loki
   \item Nutzung von Regelsätzen in Loki für die Analysierung der \gls{ssh} Logdateien
   \item Graphische Darstellung der Ergebnissen in Grafana mit den in Loki verwendeten Regelsätzen
\end{enumerate}
}

Unser letztes Ziel besteht darin, Warnmeldungen für potenzielle Angriffe mithilfe der Ergebnisse von Loki zu generieren. Grafana kann sowohl intern mit der Funktionalität \quotes{Alerting} als auch extern mit \glsplural{plugin}, wie \textbf{Alertmanager}, Warnmeldungen generieren. Der zweite kann Daten von \gls{prometheus}, \gls{cortex} und \gls{mimir} als Datenquelle verwenden \citep{Grafana_Alertmanager} und kann Daten von beliebigen \glsplural{Endpoint} empfangen. Die Regelsätze des Alertmanagers haben folgendes Muster:

{\setstretch{1.0}
\begin{Verbatim}[frame=single]
# Warnmeldungen können in beliebigen Gruppen kategorisiert werden. Diese
können von den Nutzern entsprechend ihrer Anforderungen und Bedürfnisse 
definiert werden.
groups:

      # Ab diesem Punkt beginnen wir mit der Definition der Regelsätze 
      für die Erkennung von Warnmeldungen. Diese umfassen:
   - name: example
     rules:
    - alert: HighRequestLatency

      # LogQL-Regelsätze für die Erkennung der Warnmeldung, welche die 
      in den vorherigen Schritten definierten Abfragen verwenden.
      expr: job:request_latency_seconds:mean5m{job="SSH_LOGS"} > 0.5
      for: 10m
       labels:
         severity: page
       annotations:
         summary: High request latency
\end{Verbatim}
}

%verbatim comments
%lst listing

Grafana hat auch ein eigenes internes Tool, um Warnmeldungen zu konfigurieren: \textbf{Alerting}. In dieser Arbeit versuchen wir unser Warnmeldungs-System mithilfe dieses Tools aufzubauen.

Die Warnmeldungen können direkt in der \gls{GUI} von Grafana konfiguriert werden. Dazu folgt man den folgenden Schritten \citep{Grafana_alerting}:

{\setstretch{1}
\begin{enumerate}[noitemsep]
   \item Name der Regel
   \item Regelsätze in \gls{logql}
   \item Definition von Gruppen für jede Art von Warnmeldung. Gruppen können später verschiedenen Einstellungen zugewiesen werden, wie z.B. Benachrichtigungen und Inhalte.
   \item Informationen über die Warnmeldung, wie eine eindeutige ID und eine Beschreibung. Der Nutzer kann diese Felder so definieren, wie es notwendig ist.
   \item Benachrichtigung der Zielgruppe, die diesen Fall später bearbeiten wird.
   \item Labels zur besseren Organisation der Warnmeldungen.
   \item Konfiguration von E-Mail in Grafana für die Weiterleitung der Warnmeldungen.
\end{enumerate}
}

Für unseren ersten Test erstellen wir Warnmeldungen für fehlgeschlagene Anmeldeversuche. Wir haben die oben genannten Elemente definiert und die folgenden Regelsätze verwendet \citep{VoidQuark_sshlogs}:

{\setstretch{1.0}
\begin{Verbatim}[frame=single]
# (A) Anzahl von fehlgeschlagenen Anmeldeversuche für existierenden
Benutzernamen:
sum by (username) (count_over_time({$label_name=~"$label_value", 
job=~"$job", instance=~"$instance"} |="sshd[" |~": Invalid|: 
Connection closed by authenticating user|: Failed .* user" | 
pattern '<_> user <username> <_> port' | __error__="" 
[$__interval]))

# (B) Anzahl von Fehlgeschlagenen Anmeldeversuche für nicht 
existierenden Benutzernamen:
sum by (username) (count_over_time({$label_name=~"$label_value", 
job=~"$job", instance=~"$instance"} |="sshd[" |=": Failed" !~"invalid 
user" | pattern '<_> for <username> from <_> port' | __error__=""
[$__interval]))

# Wenn die Anzahl von (A) oder von (B) größer als fünf ist, dann wird
die Warnmeldung als E-Mail an dem Ziel geschickt.
\end{Verbatim}
}

% discard lines with Server Ansible - https://grafana.com/docs/loki/latest/logql/log_queries/

Im Anhang \ref{appendix:Warnmedungskonfiguration} befindet sich die Konfigurationsdatei für unsere Warnmeldung. Nachdem alles korrekt konfiguriert wurde, haben wir die folgende E-Mail erhalten:

\begin{figure}[H]
   \centering
   \includegraphics[width=0.7\textwidth]{assets/GrafanaWarnmeldung.png}
   \caption[E-Mail Warnmeldung von Grafana]
   {E-Mail Warnmeldung von Grafana \\Quelle: Eigene Quelle und \citep{Grafana_alerting}}
   \centering
\end{figure}

%Das Alerting-Tool von Grafana bietet keine direkte Integration zu einem \gls{IDS}, \gls{IPS}, \gls{SIEM} oder einer \gls{API} an. Die Kommunikation mit solchen \glsplural{Endpoint} lässt sich jedoch mithilfe von \textbf{\gls{webhook}} konfigurieren \citep{Grafana_Notifications}.

% enabled = true
% host = smtp.gmail.com:587
% user = alertsgrafanaBA@gmail.com
% # If the password contains # or ; you have to wrap it with triple quotes. Ex """#password;"""
% password = "sexpjkbdsdwrkgqm"