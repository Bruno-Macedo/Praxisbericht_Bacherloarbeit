\begin{abstract}

    \begin{center}
        \textbf{Abstract}
    \end{center}

%This thesis has as goal to use an \gls{opensource} similar \glsfirst{SIEM} tool to monitor security events. Most of the existing \glsplural{SIEM} solutions are either \gls{Proprietary} or have only limited free features. For that we decide to implement our monitoring system using Grafana and its integrated elements: Promtail, Loki and Alerting. Grafana is primarly used to generate graphics based on the user's input. In our study, we used \gls{ssh} log files as input. The files were extracted by Promtail from the \glsplural{Endpoint} and sent to Loki which aggregates them and filters their content based on the rules defined to identify possible cyberatacks against a \gls{ssh} Server. Once the searched information was extracted, it was sent to Grafana to generate an visually overview about the \gls{ssh} connections. Eventually we used the tool Alerting to send notifications about pottentially attacks identified in our rules. The recognition of the possible attack and the ruleset used was based on the description provided by \gls{mitre} Matrix. The combined use of the aforementioned tools proved to be realiable, affordable and usefull to detect static base attacks. The main burdens of using these tools as a replacement to a \gls{SIEM} solution would be: the properly definition of the ruleset used to read and extract information about cyberattacks from the log files; and adapting those rules to scerarios where attacks have more dynamic flows.

%The goal of this thesis is to use an \gls{opensource}, similar tool to a \glsfirst{SIEM} system for monitoring security events. Most of the existing \glsplural{SIEM} solutions are either proprietary or have limited free features. Therefore, we decided to implement our monitoring system using Grafana and its integrated elements: Promtail, Loki, and Alerting. Grafana is primarily used to generate graphics based on user input. In our study, we used \gls{ssh} log files as input. Promtail extracted the files from the \glsplural{Endpoint} and sent them to Loki, which aggregated and filtered their content based on defined rules to identify possible cyberattacks against an \gls{ssh} server. Once the information was extracted, Grafana was used to generate a visual overview of the \gls{ssh} connections. We used the Alerting tool to send notifications about potential attacks identified by our rules. The recognition of possible attacks and the ruleset used was based on the descriptions provided by the \gls{mitre} Matrix. The combined use of these tools proved to be reliable, affordable, and useful in detecting static-based attacks. The main challenges of using these tools as a replacement for a \gls{SIEM} solution would be properly defining the ruleset used to read and extract information about cyberattacks from log files and adapting those rules to scenarios where attacks have more dynamic flows.

The aim of this thesis is to develop a reliable, cost-effective solution for monitoring security events by utilizing an \gls{opensource}, \gls{SIEM}-like tool. Since many existing \gls{SIEM} solutions are either proprietary or offer limited free features, we chose to use Grafana and its integrated tools - Promtail, Loki, and Alerting - to create our monitoring system. Grafana is primarily used to generate customizable graphics based on user input, and in our study, we used \gls{ssh} log files as input. Promtail extracted the files from \glsplural{Endpoint} and sent them to Loki, which used defined rules to aggregate and filter the content in order to identify possible cyberattacks against an \gls{ssh} server. Once the information was extracted, Grafana was used to provide a visual overview of the \gls{ssh} connections. Additionally, we employed the Alerting tool to send notifications about potential attacks identified by our rules. The ruleset we used to recognize potential attacks and the descriptions of these attacks were based on the \gls{mitre} Matrix. We found that the combined use of these tools was reliable, affordable, and useful for detecting static-based attacks. The main challenges in using these tools as a replacement for a \gls{SIEM} solution are properly defining the ruleset used to read and extract information about cyberattacks from log files and adapting those rules to scenarios where attacks have more dynamic flows.


\vspace{3cm}
\textbf{Keywords: Monitoring Tool, Grafana Loki Cyberattacks,\gls{SIEM}}


\end{abstract}