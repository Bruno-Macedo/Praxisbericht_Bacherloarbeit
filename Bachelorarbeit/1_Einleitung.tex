\section{Einleitung}

Cloud-Computing entwickelte sich in den letzten Jahren rasch. Die Idee von einer Auslagerung von Rechenzentren zog viele Firmen an, ihre gesamte Struktur umzubauen \citep{Alalawi_CCR}. 

Eine Cloud-Umgebung bietet Flexibilität und Skalierbarkeit an, ohne dass mehr Investitionen in Hardware-Komponenten oder in physikalische Elementen benötigt wird \citep{Obrutsky_CSAD}. Die Firmen können schnell entscheiden und implementieren, ob sie mehr oder weniger Ressource einzusetzen brauchen. Zusätzlich ist die gesamte Verwaltung des Rechenzentrums unkomplizierter und kostengünstiger, da die sie sich auf ihre eigenen Produkte konzentrieren können, während der Cloud-Anbieter dazu pflichtig ist, die Struktur verfügbar und sicher laufen zu lassen.

Besonders kleine und mittlere Unternehmen können von der Umwandeln zum Cloud-Struktur profitieren \citep{Khalid_CCAI}. Unter den vielen Vorteilen spielt die niedrigere Investitionen im Vergleich zu physikalischen Struktur eine wichtige Rolle, da die Firmen sich auf ihre Endprodukte fokussieren können \citep{Donnery_CCSB}.

Von einer Seite gehört die Skalierbarkeit von Cloud-Computing zu einem des wichtigsten Vorteil dieses Struktur, von anderer Seite kann die Verwaltung von zuwachsenden viele Ressource unübersichtlich sein. Mit dieser Arbeit wollen wir die Verwaltung von wachsenden Cloud-Computer in einer Azure Umgebung verstehen und verbessern, so dass wir einen optimierten Vorschlag für die Implementierung anbieten können. Damit wir zu unserem Ziel kommen, wird sich diese Arbeit mit folgenden Themen auseinandersetzen:

\begin{itemize}
   \item Definition von Cloud-Computing
   \item Arten von existierenden Cloud-Umgebung
   \item Verwaltung von Cloud-Infrastruktur
   \item Sicherheitsstandards in dem Cloud-Computing Umgebung
\end{itemize}