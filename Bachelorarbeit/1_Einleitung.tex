\section{Einleitung}

Der Netzwerkverkehr wuchs in den letzten 20 Jahre mehr als 100\% \citep{Roser_I}. Das Internet wird heutzutage für fast alle unsere alltägliche Tätigkeit verwendet:  Sozialenetzwerke, Video und Audio-Streaming, Einkauf, behördliche Angelegenheit und viele andere. So viel Verkehr generiert eine unermessliche Menge von Daten, die alle mögliche Inhalte beinhalten, von unschuldigen Anfragen nach dem eigenen Kontostand bis zu der Ausführung von bösewichten Anfragen, um Systemen lahmzumachen. Um das erste von der zweiten zu Unterscheiden verwenden vielen Firmen die sogennanten \glsfirst*{SIEM}.

Die \glsfirst{NIST} definiert \acrshort{SIEM} als Anwendung, die dafür zustäntig ist, Sicherheitsdaten von anderen Systemen zu sammeln und diese verständlich und lesbar als Information zu liefern. Mit diesem Ergebnis können Aktionen und durchgeführt werden können \citep{NIST_SIEM}. Die Bewertung dieser Daten spielt eine wesentliche Rolle bei solchen Anwendungen, da es entscheidend ist, ob es um eine oder viele normale Anfrage oder um einen \gls{Cyberangriff} geht.

In diesem Projekt wollen wir über eine existierende \gls{Open Source} \gls{SIEM}-Anwendung recherchieren und ihre Extrahierung und Bewertung von Daten analysieren, sodass wir schließlich eine einige Lösung für die Identifizierung von spezifische \glsplural{Cyberangriff} entwerfen können.

Diese Arbeit wird in folgende Teile geteilt:

\begin{itemize}[noitemsep]
   \item Beschreibung von existierenden \glspl{SIEM} und Vergleich zwischen privaten Anbieter und eine \gls{Open Source} Lösungen (Alienvault OSSIN, OpenSearch, MozDef, Wazuh )
   \item Analyse der Funktionalität einer \gls{Open Source} \gls{SIEM}
   \item Definition von zwei spezifische \glsplural{Cyberangriff}
   \item Empfang und Bearbeitung der Daten von den vorher beschriebenen Angriffe
   \item Entwicklung einer Regel für die Erkennung eines \gls{Cyberangriff}
   \item Analyse und Bewertung der Arbeit
\end{itemize}
